\paragraph{Choix de conception} L'interface graphique a été réalisé avec un souci de clareté et d'intuitivité. L'objectif de l'interface graphique est d'être aussi simple d'utilisation que possible. L'utilisateur doit pouvoir se souvenir de la procédure à suivre lorsqu'il utilise plusieurs fois la même fonctionnalité afin de profiter d'une expérience la plus homogène possible. En accord avec les \emph{Interaction overview diagrams}, chaque application suit la même structure (mais ne possède pas le même contenu, évidemment) : une partie "authentification" et une partie "accès à l'application". L'authentification permet de se connecter et de s'inscrire, tandis que l'accès à l'application requiert que l'utilisateur soit connecté et lui permet d'accéder à toutes les fonctionnalités disponibles. Durant cette partie de l'exécution, l'interface graphique de chaque application propose un menu principal, qui montre à l'utilisateur les fonctionnalités principales de l'application. L'interface graphique tente au mieux de suivre une exécution qui pourrait s'apparenter à un abre : on part d'un point principal et on parcourt les branches jusqu'à arriver à une feuille. Cette structure peut aussi être associée au design pattern \emph{Composition}. En respectant ce squelette, l'utilisateur ne tourne jamais en rond et peut plus facilement se repérer parmis les menus. Un exemple de ce qui ne sera pas implémenté car cela ne respecterait pas cette structure serait un bouton \emph{Back to main menu} dans chaque fenêtre qui amènerait instantanément l'utilisateur au menu principal, ou un bouton \emph{Notifications} dans chaque fenêtre qui amènerait instantanément l'utilisateur à la fenêtre \emph{Notifications}.