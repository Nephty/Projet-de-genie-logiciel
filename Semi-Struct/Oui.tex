\documentclass{article}
\newcounter{rownumbers}
\newcommand\rownumber{\stepcounter{rownumbers}\arabic{rownumbers}}
% Use utf-8 encoding for foreign characters
\usepackage[utf8]{inputenc}
% For euro symbol
\usepackage{eurosym}
% for http links
\usepackage[hidelinks]{hyperref}

% Setup for fullpage use
\usepackage{fullpage}
%\usepackage{graphicx}
%\usepackage{subfig}
%\usepackage{float}
%\usepackage{listings}

\usepackage[francais]{babel}

\usepackage{times}
%\usepackage{rotate}
%\usepackage{lscape}

\usepackage{color}
\usepackage{placeins}

\usepackage{forest}
\usepackage{amsmath}
\usepackage{amssymb}
\usepackage{algpseudocode}
\usepackage{graphicx}
\usepackage{subfiles}
\usepackage{geometry}


%-----Commandes pour page de garde UMons
\usepackage[fs]{umons-coverpage}
\umonsAuthor{
	Augustin \textsc{Houba} \\
	Arnaud \textsc{Moreau} \\
	Cyril \textsc{Moreau} \\
	François \textsc{Vion}
}
\umonsTitle{Projet de Génie Logiciel}
\umonsSubtitle{S-INFO-015}
\umonsDocumentType{Rapport de modélisation\\
7 décembre 2021}
\umonsSupervisor{ \\
	Prof. Tom \textsc{Mens}\\
} 
% The date (or academic year)
\umonsDate{Année académique 2021-2022}

\begin{document}
%\tikzset
%	every label/.append style={font=\scriptsize},
%	my edge labels/.style={font=\scriptsize},
%	dominant/.append style={label=below:$dominant$},
%}
%\frontmatter          % for the preliminaries
%\pagestyle{headings}  % switches on printing of running heads
%\mainmatter              % start of the contributions

%\umonsCoverPage

\subfile{resources/Titlepage.tex}

\newpage

\tableofcontents

\newpage

%Use cases commun aux deux diagrammes.
\section{Partie commune}
\subsection{Commun aux 2 diagrams}
\subsubsection{Se connecter}
   \begin{table}[!h]
         \begin{tabular}{|c|p{10cm}|}
            \hline
            Acteur : & Utilisateur et institution financière   \\
            \hline
            Description : & Permet à l'utilisateur de s'identifier à l'application et d'accèder à son compte \\
            \hline
            Préconditions : &Le client n'est pas encore identifié   \\
            \hline
            PostConditions : &Le client a accès à ses portefeuilles financiers ainsi qu'à toutes les fonctionnalités de l'application   \\
            \hline
            Scénario principal : & 
                  \begin{itemize}
                     \item[1.] Le client donne son login et son mot de passe.
                     \item[2.] Une demande est envoyée au serveur afin de savoir si les données sont correctes.
                     \item[3.] Le serveur envoie une confirmation de connexion.
                     \item[4.] Le client est connecté. 
                  \end{itemize}     \\
            \hline
            Scénario alternatif : &
                  \begin{itemize}
                     \item[3b.] Une erreur est survenue ou le client n'a pas introduit les bons identifiants.
                     \begin{itemize}
                        \item[3b1.] Un message d'erreur est affiché à l'écran. 
                        \item[3b2.] L'utilisateur est invité à réintroduire ses identifiants de connexion. 
                     \end{itemize} 
                  \end{itemize}      \\
            \hline
            Trigger : & Lorsque l'utilisateur clique sur le bouton se connecter.    \\
            \hline
            Fréquence d'utilisation : & A chaque lancement de l'application et lorsque l'utilisateur souhaite changer de compte.     \\
            \hline
         \end{tabular}
   \end{table}

   \newpage

\subsubsection{Se déconnecter}
   \begin{table}[h]
         \begin{tabular}{|c|p{10cm}|}
            \hline
            Acteurs principaux& Utilisateur et institution financière \\
            \hline
            Description&Permet au client de se déconnecter du compte auquel il est connecté actuellement \\
            \hline
            Préconditions&\^Etre connecté à un compte utilisateur.\\
            \hline
            PostConditions&L'interface utilisateur affiche la méthode de connexion et le client n'a plus accès à son compte sans s'être reconnecté.\\
            \hline
            Scénario principal& 
                  \begin{enumerate}
                     \item La session de l'utilisateur se ferme.
                     \item L'utilisateur se trouve sur la page de connexion.
                  \end{enumerate}     \\
            \hline
            Scénario alternatif&/\\
            \hline
            Trigger&L'utilisateur appuie sur le bouton permettant de se déconnecter.\\
            \hline
            Fréquence d'utilisation&Assez rare\\
            \hline
         \end{tabular}
   \end{table}

\newpage

\subsubsection{Changer de mot de passe}
\begin{table}[h]
   \begin{tabular}{|c|p{10cm}|}
      \hline
      Acteur principaux&Client et institution financière. \\
      \hline
      Description&Permet à l'utilisateur de changer le mot de passe de son compte.\\
      \hline
      Préconditions&L'utilisateur est connecté à son compte et a rempli les champs pour son nouveau mot de passe.\\
      \hline
      PostConditions&Le mot de passe a été modifié.\\
      \hline
      Scénario principal& 
            \begin{enumerate}
               \item Le cliet donne son mot de passe actuel ainsi que le nouveau mot de passe.
               \item Une demande est envoyée au serveur afin qu'il vérifie que le mot de passe actuel est correct.
               \item 
            \end{enumerate}     \\
      \hline
      Scénario alternatif A&
               \begin{itemize}
                  \item[3b1] Le mot de passe est incorrect.
                  \item[3b2] Une erreur s'affiche à l'écran de l'utilisateur.
                  \item[3b3] L'utilisateur est invité à réintroduire son mot de passe. 
               \end{itemize}      \\
      \hline
      Scénario alternatif B& 
               \begin{itemize}
                  \item[5b1] Une erreur s'est produite.
                  \item[5b2] Une erreur s'affiche à l'écran de l'utilisateur.
                  \item[5b3] L'utilisateur revient à l'écran de changement de mot de passe.   
               \end{itemize} \\
      \hline
      Trigger&Lorsque l'utilisateur appuie sur le bouton de changement de mot de passe.\\
      \hline
      Fréquence d'utilisation&Assez rare\\
      \hline
   \end{tabular}
\end{table}

\newpage

\subsubsection{Changer la langue}

\begin{table}[h]
   \begin{tabular}{|c|p{10cm}|}
      \hline
      Acteurs principaux&Le client et l'institution fiancière \\
      \hline
      Description&Permet de changer la langue de l'application.\\
      \hline
      Préconditions&L'application est lancée.\\
      \hline
      PostConditions&La langue de l'application est changée.\\
      \hline
      Scénario principal& 
            \begin{enumerate}
               \item L'utilisateur sélectionne sa langue favorite présente dans les fichiers de l'application.
               \item Une demande est envoyée au serveur pour modifier la langue préférée de l'utilisateur.
               \item Le serveur a modifié la langue préférée de l'utilisateur.
               \item La langue change et un message de confirmation est affiché.
            \end{enumerate}     \\
      \hline
      Scénario alternatif& / \\
      \hline
      Trigger&Lorsque l'utilisateur confirme la langue qu'il souhaite avoir.\\
      \hline
      Fréquence d'utilisation&Moyenne\\
      \hline
      Note&Si l'utilisateur se connecte à son compte et que sa langue préférée n'est pas installée, 
      un message est affiché et l'anglais est choisi par défaut.\\
      \hline
   \end{tabular}
\end{table}

\newpage

\subsubsection{Créer un compte}
\begin{table}[h]
   \begin{tabular}{|c|p{10cm}|}
      \hline
      Acteurs principaux&Client et institution financière\\
      \hline
      Description&L'utilisateur crée un compte afin d'utiliser l'application.\\
      \hline
      Préconditions&L'application est lancée et l'utilisateur n'est pas connecté à un compte.\\
      \hline
      PostConditions&Le compte de l'utilisateur est créé et il est connecté.\\
      \hline
      Scénario principal& 
            \begin{enumerate}
               \item L'utilisateur entre les informations dans les champs correspondants et confirme.
               \item Une vérification sur les informations est effectuée.\linebreak
               (exemple : adresse mail ou username déjà utilisé)
               \item Une demande de création de compte est envoyée au serveur.
               \item Le serveur crée le compte et envoie une confirmation.
               \item L'utilisateur se retrouve sur l'écran d'acceuil, connecté à son compte.
            \end{enumerate}     \\
      \hline
      Scénario alternatif A&
            \begin{itemize}
               \item[2a1] Un des identifiants est déjà utilisé. 
               \item[2a2] Un message d'erreur apparaît à l'écran.
               \item[2a3] L'utilisateur est invité à changer ses identifiants pour créer son compte.  
            \end{itemize}      \\
      \hline
      Scénario alternatif B&
            \begin{itemize}
               \item[4b1] Une erreur s'est produite sur le serveur.
               \item[4b2] Une erreur apparaît à l'écran. 
            \end{itemize} \\
      \hline
      Trigger&Lorsque l'utilisateur appuie sur le bouton permettant de créer un compte.\\
      \hline
      Fréquence d'utilisation&Moyenne(lorsqu'un nouvelle utilisateur lance l'application)\\
      \hline
   \end{tabular}
\end{table}

\newpage

\subsection{Application client}
\subsubsection{Introduire une demande}
\begin{table}[h]
   \begin{tabular}{|c|p{10cm}|}
      \hline
      Acteur principal&Le client\\
      \hline
      Description&Le client introduit une demande à une institution (demande d'accès aux virements, d'ajout de co-titulaire,...)    \\
      \hline
      Préconditions&Le client est connecté à un compte.\\
      \hline
      PostConditions&La demande du client est envoyée à l'insitution financière concernée et le client est averti que l'action s'est bien déroulée.\\
      \hline
      Scénario principal& 
            \begin{enumerate}
               \item Le client rempli le formulaire selon le type de demande qu'il souhaite envoyer
               \item La demande est envoyée sur le serveur afin de la faire parvenir à l'institution financière concernée.
               \item Le client est notifié de l'action effectuée.
            \end{enumerate}     \\
      \hline
      Scénario alternatif&
            \begin{itemize}
               \item[3b1] Une erreur s'est produite lors de la gestion de la requête par le serveur.
               \item[3b2] Un message d'erreur s'affiche à l'écran.  
            \end{itemize}      \\
      \hline
      Trigger&Le client appuie sur le bouton permettant d'introduire une demande.\\
      \hline
      Fréquence d'utilisation&Assez souvent.\\
      \hline
   \end{tabular}
\end{table}

\newpage

\subsubsection{Afficher les notifications/demandes}
\begin{table}[h]
   \begin{tabular}{|c|p{10cm}|}
      \hline
      Acteur principal&Le client\\
      \hline
      Description&Permet au client de voir la liste de ses notifications ainsi que ses demandes\\
      \hline
      Préconditions&Le client est connecté à un compte\\
      \hline
      PostConditions&La liste est affichée\\
      \hline
      Scénario principal& 
            \begin{enumerate}
               \item Une demande est envoyée au serveur afin d'avoir accès à la liste du client.
               \item Le serveur renvoie la liste du client concerné.
               \item Le client a accès à ses notifications et demandes.
            \end{enumerate}     \\
      \hline
      Scénario alternatif&
            \begin{itemize}
               \item[2b1] Une erreur s'est produite.
               \item[2b2] Le client est notifié de l'erreur.
            \end{itemize}      \\
      \hline
      Trigger&Le client appuie sur le bouton permettant d'avoir accès à ses notifications/demandes\\
      \hline
      Fréquence d'utilisation&Moyenne\\
      \hline
   \end{tabular}
\end{table}

\newpage

\subsubsection{Voir la liste des portefeuilles}
\begin{table}[h]
      \begin{tabular}{|c|p{10cm}|}
         \hline
         Acteur principal&Le client\\
         \hline
         Description&Affiche la liste des portefeuilles auxquels le client a accès.\\
         \hline
         Préconditions&L'application est lancée et le client est connecté à un compte \\
         \hline
         PostConditions&La liste des portefeuilles est affichée.\\
         \hline
         Scénario principal& 
               \begin{enumerate}
                  \item Une requête est envoyée au serveur afin d'avoir accès à la liste des portefeuilles du client.
                  \item Le serveur renvoie cette liste.
                  \item La liste est affichée à l'écran du client.
               \end{enumerate}     \\
         \hline
         Scénario alternatif&
               \begin{itemize}
                  \item[2b1] Une erreur est survenue lors de la gestion de la requête par le serveur.
                  \item[2b2] Un message d'erreur est affiché. 
               \end{itemize}\\
         \hline
         Trigger&Le client appuie sur le bouton permettant de lister ses portefeuilles.\\
         \hline
         Fréquence d'utilisation&Très souvent.\\
         \hline
      \end{tabular}
\end{table}

\newpage

\subsubsection{Sélectionner un produit}
\begin{table}[h]
      \begin{tabular}{|c|p{10cm}|}
         \hline
         Acteur principal&Le client \\
         \hline
         Description&Un client sélectionne un produit.\\
         \hline
         Préconditions&Le client est connecté à l'application et se trouve devant la liste des portefeuilles.\\
         \hline
         PostConditions&Le client a maintenant accès aux détails du produit (historique,...)\\
         \hline
         Scénario principal& 
               \begin{enumerate}
                  \item Le client sélectionne le produit qu'il souhaite voir
                  \item Une demande est envoyée au serveur pour avoir accès à l'historique du produit
                  \item Le serveur revnoie l'historique.
                  \item Le client se retrouve sur la page de selection de produit et l'historique du compte est affiché
               \end{enumerate}     \\
         \hline
         Scénario alternatif&
               \begin{itemize}
                  \item[3b1] Une erreur s'est produite.
                  \item[3b2] L'utilisateur est notifié de l'erreur.  
               \end{itemize}\\
         \hline
         Trigger&Le client clique sur un produit dans la liste de portefeuilles\\
         \hline
         Fréquence d'utilisation&Assez souvent\\
         \hline
      \end{tabular}
\end{table}

\newpage

\subsubsection{Modifier un produit}
\begin{table}[h]
      \begin{tabular}{|c|p{10cm}|}
         \hline
         Acteur principal&Le client\\
         \hline
         Description&Choix d'activer ou de désactiver un produit\\
         \hline
         Préconditions&Le client est connecté à son compte et a sélectionné un produit.\\
         \hline
         PostConditions&Le compte du client a été activé ou désactivé.\\
         \hline
         Scénario principal& 
               \begin{enumerate}
                  \item Le client choisi d'activer/désactiver un compte.
                  \item Une demande de modification est envoyée au serveur.
                  \item Le serveur renvoie une confirmation.
                  \item Un message de succès est affiché.
               \end{enumerate}     \\
         \hline
         Scénario alternatif&
               \begin{itemize}
                  \item[3b1] Une erreur est survenue lors du traitement de la demande par le servuer
                  \item[3b2] Un message d'erreur est affiché à l'écran de l'utilisateur.  
               \end{itemize}\\
         \hline
         Trigger&Le client clique sur un bouton de modification de compte.\\
         \hline
         Fréquence d'utilisation&Moyenne\\
         \hline
      \end{tabular}
\end{table}

\newpage

\subsubsection{Effectuer un virement}
\begin{table}[h]
      \begin{tabular}{|c|p{10cm}|}
         \hline
         Acteur principal&Le client\\
         \hline
         Description&Effectuer une transaction entre deux comptes bancaires.\\
         \hline
         Préconditions&Être connecté à un compte et effectuer le virement via un compte autorisé.\\
         \hline
         PostConditions&Le virement a bien été enregistré et est prêt à être envoyé.\\
         \hline
         Scénario principal& 
               \begin{enumerate}
                  \item L'utilisateur donne le numéro de compte du bénéficiaire
                  \item Le client donne le monde du virement qu'il souhaite effectuer.
                  \item Une demande est envoyée au serveur.
                  \item Une confirmation est reçue par le client.
               \end{enumerate}     \\
         \hline
         Scénario alternatif&
               \begin{itemize}
                  \item[4b1] Une erreur s'est produite lors de la gestion de la demande par le serveur.
                  \item[4b2] Le client est notifié de l'erreur.  
               \end{itemize}\\
         \hline
         Trigger&Lorsque le client clique sur le bouton de virement.\\
         \hline
         Fréquence d'utilisation&Souvent\\
         \hline
      \end{tabular}
\end{table}

\newpage

\subsubsection{Voir l'historique d'un ou plusieurs comptes}
\begin{table}[h]
      \begin{tabular}{|c|p{10cm}|}
         \hline
         Acteur principal&Le client\\
         \hline
         Description&Affiche l'historique du/des comptes(s) que le client a selectionné sur la période sélectionnée.\\
         \hline
         Préconditions&Le client a selectionné au moins un produit\\
         \hline
         PostConditions&L'historique est affiché.\\
         \hline
         Scénario principal& 
               \begin{enumerate}
                  \item Une demande est envoyée au serveur afin d'avoir l'historique des produits sélectionnés.
                  \item Le serveur renvoie les données demandées.
                  \item L'application affiche les données et peut ensuite changé la méthode d'affichage et la période d'affichage.
               \end{enumerate}     \\
         \hline
         Scénario alternatif&
               \begin{itemize}
                  \item[2b1] Une erreur survient lors de la gestion de la demande par le serveur.
                  \item[2b2] Le client est invité a réintroduire sa demande ou de contacter un administrateur.  
               \end{itemize} \\
         \hline
         Trigger&Lorsque l'utilisateur appuie sur le bouton permettant d'accéder à l'historique.\\
         \hline
         Fréquence d'utilisation&Moyenne\\
         \hline
      \end{tabular}
\end{table}

\newpage

\subsubsection{Exporter l'historique}
\begin{table}[h]
      \begin{tabular}{|c|p{10cm}|}
         \hline
         Acteur principal&Le client\\
         \hline
         Description&Permet d'exporter l'historique d'un ou plusieurs comptes vers des fichiers en format CSV ou JSON\\
         \hline
         Préconditions&Être connecté à un compte et être devant la visualisation d'historique.\\
         \hline
         PostConditions&Le fichier est exporté sur la machine de l'utilisateur dans le format choisi\\
         \hline
         Scénario principal& 
               \begin{enumerate}
                  \item L'application demande au client de choisir sous quel format il souhaite l'exporter.
                  \item Le fichier est exporté sur la machine du client sous le format sélectionné.
               \end{enumerate}     \\
         \hline
         Scénario alternatif&
         \begin{itemize}
            \item[2b1] Une erreur s'est produite.
            \item[2b2] Le client est informé de l'erreur.  
         \end{itemize}\\
         \hline
         Trigger&Le client clique sur le bouton permettant d'exporter les données.\\
         \hline
         Fréquence d'utilisation&Assez rare/moyenne\\
         \hline
      \end{tabular}
\end{table}

\newpage
\subsection{Institution financière}

\subsubsection{Afficher la liste de ses clients}
\begin{table}[h]
   \begin{tabular}{|c|p{10cm}|}
      \hline
      Acteur principal&Institution financière\\
      \hline
      Description&Liste les clients de l'institution fiancière.\\
      \hline
      Préconditions&Être connecté à un compte d'une institution financière.\\
      \hline
      PostConditions&La liste des clients de l'institution est affichée\\
      \hline
      Scénario principal& 
            \begin{enumerate}
               \item Une demande est envoyée au serveur pour connaître la liste des clients.
               \item La liste des clients est renvoyées par le serveur
               \item La liste est affichée.
            \end{enumerate}     \\
      \hline
      Scénario alternatif&
            \begin{itemize}
               \item[2b1] Une erreur est survenue.
               \item[2b2] L'utilisateur est notifié de l'erreur.  
            \end{itemize}      \\
      \hline
      Trigger&L'utilisateur clique sur le bouton permettant de lister les clients de l'institution financière.\\
      \hline
      Fréquence d'utilisation&Assez souvent.\\
      \hline
   \end{tabular}
\end{table}

\newpage

\subsubsection{Supprimer un client}
\begin{table}[h]
   \begin{tabular}{|c|p{10cm}|}
      \hline
      Acteur principal&Institution fiancière\\
      \hline
      Description&Permet à l'institution de supprimer un client de leur liste.\\
      \hline
      Préconditions&Avoir affiché la liste de clients.\\
      \hline
      PostConditions&Le client est supprimé de la liste des clients de l'institution.\\
      \hline
      Scénario principal& 
            \begin{enumerate}
               \item L'institution choisi le client à supprimer
               \item Une demande de suppression est envoyée au serveur.
               \item Le serveur supprime le client de la liste des clients de l'institution.
               \item Une notification est envoyée au client concerné.
               \item Une confirmation est envoyée à l'institution.
            \end{enumerate}     \\
      \hline
      Scénario alternatif&
            \begin{itemize}
               \item[2b-4b] Une erreur est survenue
               \item[$\rightarrow$] L'utilisateur est notifie de l'erreur.  
            \end{itemize}\\
      \hline
      Trigger&L'utilisateur appuie sur le bouton permettant de supprimer un client.\\
      \hline
      Fréquence d'utilisation&Assez rare.\\
      \hline
   \end{tabular}
\end{table}

\newpage

\subsubsection{Ajouter un nouveau client}
\begin{table}[h]
   \begin{tabular}{|c|p{10cm}|}
      \hline
      Acteur principal&Institution fiancière\\
      \hline
      Description&Permet à l'institution d'ajouter un nouveau client dans leur liste\\
      \hline
      Préconditions&Avoir affiché la liste des clients de l'institution.\\
      \hline
      PostConditions&Un client est ajouté à la liste.\\
      \hline
      Scénario principal& 
            \begin{enumerate}
               \item L'utilisateur rentre les informations du client à ajouter.
               \item Une demande d'ajout est envoyée au serveur.
               \item Le serveur renvoie une confirmation.
               \item L'utilisateur est notifié que tout s'est bien passé.
            \end{enumerate}     \\
      \hline
      Scénario alternatif&
            \begin{itemize}
               \item[3b1] Une erreur s'est produite.
               \item[3b2] L'utilisateur est notifié de l'erreur.  
            \end{itemize}\\
      \hline
      Trigger&L'utilisateur appuie sur le bouton permettant d'ajouter un client.\\
      \hline
      Fréquence d'utilisation&Moyenne.\\
      \hline
   \end{tabular}
\end{table}

\newpage

\subsubsection{Sélectionner un client}
\begin{table}[h]
   \begin{tabular}{|c|p{10cm}|}
      \hline
      Acteur principal&Institution fiancière\\
      \hline
      Description&Permet de sélectionner un client dans la liste des clients de l’institution et de voir tous les produis qu’il possède dans cette institution.\\
      \hline
      Préconditions&Être connecté à un compte d’une institution financière\\
      \hline
      PostConditions&La liste des produits du client sélectionné est affichée.\\
      \hline
      Scénario principal& 
            \begin{enumerate}
               \item L’utilisateur clique sur un client à afficher
               \item Une demande est envoyée au serveur afin d’avoir accès à tous les comptes du client dans cette institution.
               \item Le serveur renvoie la liste des produits.
               \item Les données sont finalement affichées.
            \end{enumerate}     \\
      \hline
      Scénario alternatif&
            \begin{itemize}
               \item[3b1] Une erreur est survenue.
               \item[3b2] Un message d'erreur notifie l'utilisateur qu'une erreur s'est produite.  
            \end{itemize}\\
      \hline
      Trigger&L'institution financière clique sur un client\\
      \hline
      Fréquence d'utilisation&Souvent\\
      \hline
   \end{tabular}
\end{table}

\newpage

\subsubsection{Ouvrir un produit financier}
\begin{table}[h]
   \begin{tabular}{|c|p{10cm}|}
      \hline
      Acteur principal&Institution fiancière\\
      \hline
      Description&Permet d'ouvrir un nouveau produit financier pour un certain client\\
      \hline
      Préconditions&Avoir sélectionné un client\\
      \hline
      PostConditions&Le produit du client a été ouvert et le client a été notifié.\\
      \hline
      Scénario principal& 
            \begin{enumerate}
               \item L'utilisateur donne les informations permettant la création d'un produit financier (N° national du client, type du produit,...)
               \item Une demande est envoyée au serveur afin d'ajouter le produit.
               \item Le client concerné est notifié.
               \item Le serveur renvoie un confirmation de l'ajout.
               \item L'utilisateur est notifié de la réussite de la requête.
            \end{enumerate}     \\
      \hline
      Scénario alternatif A&
            \begin{itemize}
               \item[3a1] La notification n'a pas pu être envoyée.
               \item[3b2] L'action est annulée et l'utilisateur est informé de l'erreur. 
            \end{itemize}\\
      \hline
      Scénario laternatif B&
            \begin{itemize}
               \item[2a1] Une erreur s'est produite lors de l'ajout du compte.
               \item[2a2] L'utilisateur est informé.  
            \end{itemize}\\
      \hline
      Trigger&L'utilisateur clique sur le bouton permettant d'ouvrir un compte.\\
      \hline
      Fréquence d'utilisation&Assez rare.\\
      \hline
   \end{tabular}
\end{table}

\newpage

\subsubsection{Clôturer un produit financier}
\begin{table}[h]
   \begin{tabular}{|c|p{10cm}|}
      \hline
      Acteur principal&Institution fiancière\\
      \hline
      Description&Permet de clôturer un produit existant\\
      \hline
      Préconditions&Avoir selectionné un client.\\
      \hline
      PostConditions&Le produit est supprimé et le client est notifié.\\
      \hline
      Scénario principal& 
            \begin{enumerate}
               \item L'institution selectionne le produit à clôturer.
               \item Une demande de clôture est envoyée au serveur.
               \item Le client concerné est notifié
               \item Le serveur envoie une confirmation de la clôture du produit.
               \item L'utilisateur est informé de la bonne clôture du produit.
            \end{enumerate}     \\
      \hline
      Scénario alternatif A&
            \begin{itemize}
               \item[2b1] Une erreur s'est produite lors de la cloture du produit.
               \item[2b2] L'utilisateur est notifié. 
            \end{itemize}\\
      \hline
      Scénarion alternatif B&
            \begin{itemize}
               \item[3b1] La notification n'a pas été envoyée.
               \item[3b2] L'action est annulée et l'utilisateur est informé de l'erreur.  
            \end{itemize}\\
      \hline
      Trigger&L'utilisateur clique sur le bouton permettant de clôturer un compte.\\
      \hline
      Fréquence d'utilisation&Assez rare.\\
      \hline
   \end{tabular}
\end{table}

\newpage

\subsubsection{Afficher la liste des produits financiers de l'insitution}
\begin{table}[h]
   \begin{tabular}{|c|p{10cm}|}
      \hline
      Acteur principal&Institution fiancière\\
      \hline
      Description&Affiche la lsite des tous les produits financiers ouvert dans cette institution.\\
      \hline
      Préconditions&Être connecté à un compte d'un institution.\\
      \hline
      PostConditions&Avoir la liste des données demanée à l'écran.\\
      \hline
      Scénario principal& 
            \begin{enumerate}
               \item L'utilisateur choisi les critère d'affichage des données.
               \item Une demande est envoyée au serveur afin de récupérer cette liste.
               \item Le serveur renvoie les données
               \item Les données sont affichées.
            \end{enumerate}     \\
      \hline
      Scénario alternatif&
            \begin{itemize}
               \item[3b1] Une erreur s'est produite.
               \item[3b2] L'utilisateur est informé de l'erreur.  
            \end{itemize}\\
      \hline
      Trigger&L'utilisateur clique sur le bouton permettant d'afficher les données\\
      \hline
      Fréquence d'utilisation&Moyenne.\\
      \hline
   \end{tabular}
\end{table}

\newpage

\subsubsection{Importer données}
\begin{table}[h]
   \begin{tabular}{|c|p{10cm}|}
      \hline
      Acteur principal&Institution fiancière\\
      \hline
      Description&Permet à l'institution d'importer leurs données afin d'actualiser la base de données\\
      \hline
      Préconditions&Être connecté à un compte d'une institution\\
      \hline
      PostConditions&Les données ont été ajoutée dans la base de données.\\
      \hline
      Scénario principal& 
            \begin{enumerate}
               \item L'utilisateur donne le fichier à importer.
               \item L'application vérifie que les données sont au bon format et vérifie que ce sont bien les données de l'insitution.
               \item Une demande d'ajout est envoyé au serveur
               \item Le serveur renvoie une confirmation.
               \item L'utilisateur est informé que tout s'est bien passé.
            \end{enumerate}     \\
      \hline
      Scénario alternatif A&
            \begin{itemize}
               \item[2a1] Les données ne sont pas au bon format
               \item[2a2] L'utilisateur est invité à changé le fichier qu'il souhaite envoyer.  
            \end{itemize}      \\
      \hline
      Scénario alternatif B&
            \begin{itemize}
               \item[2b1] Les données n'appartiennent pas à l'institution.
               \item[2b2] L'utilisateur est informé de l'erreur.  
            \end{itemize}\\
      \hline
      Scénario alternatif C&
            \begin{itemize}
               \item[4c1] Une erreur s'est produite sur le serveur.
               \item[4c2] L'utilisateur est informé.  
            \end{itemize} \\
      \hline
      Trigger&L'utilisateur appuie sur un bouton permettant d'importer les données.\\
      \hline
      Fréquence d'utilisation&Assez rare.\\
      \hline
   \end{tabular}
\end{table}

\newpage

\subsubsection{Exporter les données}
\begin{table}[h]
   \begin{tabular}{|c|p{10cm}|}
      \hline
      Acteur principal&Institution fiancière\\
      \hline
      Description&Permet d'exporter les données affichées.\\
      \hline
      Préconditions&Avoir affiché des données.\\
      \hline
      PostConditions&Un fichier au format CSV ou JSON est exporté sur la machine de l'utilisateur\\
      \hline
      Scénario principal& 
            \begin{enumerate}
               \item L'application demande à l'utilisateur sous quel format il souhaite exporter les données.
               \item Le fichier est exporté sur la machine de l'utilisateur.
            \end{enumerate}     \\
      \hline
      Scénario alternatif&
            \begin{itemize}
               \item[2b1] Erreur lors de la génération du fichier.
               \item[2b2] L'utilisateur est notifié de l'erreur. 
            \end{itemize}      \\
      \hline
      Trigger&Le client appuie sur le bouton permettant d'exporter les données.\\
      \hline
      Fréquence d'utilisation&Moyenne.\\
      \hline
   \end{tabular}
\end{table}

\newpage

\subsubsection{Gérer les demandes}
\begin{table}[h]
   \begin{tabular}{|c|p{10cm}|}
      \hline
      Acteur principal&Institution fiancière\\
      \hline
      Description&Permet à une institution de lister les demandes des utilisateurs en attente.\\
      \hline
      Préconditions&Être connecté à un compte d'une institution.\\
      \hline
      PostConditions&Avoir la liste des demandes en attente affichée.\\
      \hline
      Scénario principal& 
            \begin{enumerate}
               \item L'utilisateur demande d'afficher la liste des demandes.
               \item Une demande est envoyée au serveur.
               \item Le serveur renvoie la liste.
               \item La liste est affichée.
            \end{enumerate}     \\
      \hline
      Scénario alternatif&
            \begin{itemize}
               \item[3b1] Erreur lors de la requête
               \item[3b2] L'utilisateur est notifié de l'erreur.  
            \end{itemize}\\
      \hline
      Trigger&L'utilisateur appuie sur le bouton permettant d'accéder aux demandes.\\
      \hline
      Fréquence d'utilisation&Assez souvent.\\
      \hline
   \end{tabular}
\end{table}

\newpage

\subsubsection{Approuver une demande.}
\begin{table}[h]
   \begin{tabular}{|c|p{10cm}|}
      \hline
      Acteur principal&Institution fiancière\\
      \hline
      Description&Permet à une institution financière d'approuver une demande.\\
      \hline
      Préconditions&Être connecté à un compte d'une institution et avoir affiché la liste des demandes.\\
      \hline
      PostConditions&La demande a été approuvée et une notification est envoyée à l'utilisateur.\\
      \hline
      Scénario principal& 
            \begin{enumerate}
               \item L'utilisateur approuve la demande d'un client.
               \item Une notification est envoyée à l'utilisateur pour le prévenir que sa demande a été acceptée.
               \item Le serveur renvoie une confirmation que tout s'est bien passé.
               \item L'utilisateur est notifié que tout s'est bien passé.
            \end{enumerate}     \\
      \hline
      Scénario alternatif&
            \begin{itemize}
               \item[2b1] Une erreur s'est produite.
               \item[$\rightarrow$] L'utilisateur est notifié de l'erreur.  
            \end{itemize}\\
      \hline
      Trigger&L'utilisateur appuie sur le bouton permettant d'approuver une demande\\
      \hline
      Fréquence d'utilisation&Assez souvent\\
      \hline
   \end{tabular}
\end{table}

\newpage

\subsubsection{Refuser une demande.}
\begin{table}[h]
   \begin{tabular}{|c|p{10cm}|}
      \hline
      Acteur principal&Institution fiancière\\
      \hline
      Description&Permet à une institution financière de refuser une demande.\\
      \hline
      Préconditions&Être connecté à un compte d'une institution et avoir affiché la liste des demandes.\\
      \hline
      PostConditions&La demande a été refusée et une notification est envoyée à l'utilisateur.\\
      \hline
      Scénario principal& 
            \begin{enumerate}
               \item L'utilisateur approuve la demande d'un client.
               \item Une notification est envoyée à l'utilisateur pour le prévenir que sa demande a été refusée.
               \item Le serveur renvoie une confirmation que tout s'est bien passé.
               \item L'utilisateur est notifié que tout s'est bien passé.
            \end{enumerate}     \\
      \hline
      Scénario alternatif&
            \begin{itemize}
               \item[2b1] Une erreur s'est produite.
               \item[$\rightarrow$] L'utilisateur est notifié de l'erreur.  
            \end{itemize}\\
      \hline
      Trigger&L'utilisateur appuie sur le bouton permettant de refuser une demande\\
      \hline
      Fréquence d'utilisation&Assez souvent\\
      \hline
   \end{tabular}
\end{table}

\newpage

\section{Extension 2 - Moreau Cyril}
\subsection{Application client}
\subsubsection{Voir taux de conversion devise}
\begin{table}[h]
   \begin{tabular}{|c|p{10cm}|}
      \hline
      Acteur principal&Client\\
      \hline
      Description&Permet à un client de voir les taux actuels de conversions, calculer la transformations entre devise et permet de voir l'historique des taux de conversion de cette devise.\\
      \hline
      Préconditions&Être connecté à un compte client.\\
      \hline
      PostConditions&Taux de conversion de chaque devise affichée et calcul de transformation effectué.\\
      \hline
      Scénario principal& 
            \begin{enumerate}
               \item L'application envoie une demande au serveur afin de récupérer le taux de conversion d'une devise.
               \item Le serveur renvoie ce qui a été demandé.
               \item Si l'utilisateur a entré un calcul à effectuer, l'application effectue le calcul.
               \item Les taux de conversion, le résultat du calcul sont affichés ainsi que l'historique des taux de conversion sont affichés.
            \end{enumerate}     \\
      \hline
      Scénario alternatif&
            \begin{itemize}
               \item[2b1] Une erreur s'est produite lors de la requête.
               \item[2b2] L'action est annulée et un message d'erreur s'affiche.  
            \end{itemize}\\
      \hline
      Trigger&L'utilisateur appuie sur le bouton permettant d'afficher le taux de conversion de devise.\\
      \hline
      Fréquence d'utilisation&Moyenne\\
      \hline
   \end{tabular}
\end{table}

\newpage
\subsubsection{Virement internationnal}
\begin{table}[]
   \begin{tabular}{|c|p{10cm}|}
      \hline
      Acteur principal&Client\\
      \hline
      Description&Permet au client d'effectuer un virement vers un compte d'une institution se trouvant dans un autre pays.\\
      \hline
      Préconditions&Être connecté à un compte client, pouvoir effectuer des virement et effectuer ce virement vers un compte hors zone SEPA\\
      \hline
      PostConditions&Virement effectué avec les délais supplémentaires et frais supplémentaires ajoutés.\\
      \hline
      Scénario principal& 
            \begin{enumerate}
               \item L'utilisateur entre le numéro SWIFT de la banque vers laquelle il souhaite effectuer le virement ainsi que l'adresse complète du bénéficiaire.
               \item L'application converti si besoin le montant du virement vers la devise du nouveau compte.
               \item Les frais de conversion ainsi que les délais sont calculé par l'application.
               \item L'application demande à l'utilisateur qui doit payer les frais supplémentaires (Lui-même, bénéficiaire ou les deux)
               \item Une requête est envoyée au serveur afin d'ajouter cette transaction.
               \item Le serveur renvoie une confirmation de l'ajout.
               \item L'utilisateur est notifié que le virement a été accepté.
            \end{enumerate}     \\
      \hline
      Scénario alternatif A&
            \begin{itemize}
               \item[1a1] L'utilisateur n'a pas entré un swift valide et/ou une adresse valide.
               \item[1a2] Un message d'erreur invite l'utilisateur à recommencer en changeant les paramètres. 
            \end{itemize}      \\
      \hline
      Scénario alternatif B&
            \begin{itemize}
               \item[2b1] L'application n'arrive pas à accéder au taux de conversion de la devise.
               \item[2b2] Un message d'erreur est affiché et invite l'utilisateur à rééssayer plus tard.  
            \end{itemize} \\
      \hline
      Scénario alternatif C&
            \begin{itemize}
               \item[6c1] Le serveur renvoie un message d'erreur.
               \item[6c2] L'action est annulée et l'utilisateur est notifié de l'erreur  
            \end{itemize}\\
      \hline
      Trigger&L'utilisateur souhaite effectuer un virement vers un compte d'une banque se trouvant hors zone SEPA\\
      \hline
      Fréquence d'utilisation&Assez souvent.\\
      \hline
   \end{tabular}
\end{table}
\newpage

\subsection{Institution financière}
\subsubsection{Ajouter une devise à un compte}
\begin{table}[h]
   \begin{tabular}{|c|p{10cm}|}
      \hline
      Acteur principal&Institution financière\\
      \hline
      Description&Permet à un institution d'ajouter un devise à un compte multidevise.\\
      \hline
      Préconditions&Avoir selectionné un compte multidevise et être connecté à un compte d'une institution.\\
      \hline
      PostConditions&Un nouveau subAccount est ajouté avec la devise choisie.\\
      \hline
      Scénario principal& 
            \begin{enumerate}
               \item L'utilisateur choisi la devise à ajouter.
               \item Une demande est envoyée au serveur afin de créer un subaccount
               \item Le client concerné est notifié que le compte a été créé
               \item Le serveur renvoie une confirmation que tout s'est bien passé.
               \item L'utilisateur est informé que tout s'est bien passé.
            \end{enumerate}     \\
      \hline
      Scénario alternatif A& 
            \begin{itemize}
               \item[3a1] Le message de notification n'a pas pu être envoyé.
               \item[3a2] l'action est annulée et un message d'erreur est envoyé à l'utilisateur.  
            \end{itemize}     \\
      \hline
      Scénario alternatif B&
            \begin{itemize}
               \item[4b1] Une erreur s'est produite lors de la création du compte.
               \item[4b2] L'action est annulée et un message d'erreur s'affiche.  
            \end{itemize}\\
      \hline
      Trigger&Lorsque l'utilisateur appuie sur le bouton permettant d'ajouter une devise à un compte.\\
      \hline
      Fréquence d'utilisation&Moyenne.\\
      \hline
   \end{tabular}
\end{table}

\newpage

\subsubsection{Ajouter frais pour un pays}
\begin{table}[h]
   \begin{tabular}{|c|p{10cm}|}
      \hline
      Acteur principal&Institution financière\\
      \hline
      Description&Permet à une institution financière de rajouter un pays dans la liste des pays disponible pour effectuer un virement internationnal.\\
      \hline
      Préconditions&Être connecté à un compte d'une institution financière.\\
      \hline
      PostConditions&Le pays et ses frais ont été ajoutés à la liste de pays disponible pour cette institution.\\
      \hline
      Scénario principal& 
            \begin{enumerate}
               \item L'utilisateur choisi le pays à ajouter(FR-BE,...) ainsi que les frais lorsqu'un de leur client effectue un virement vers ce pays.
               \item Une requête est envoyée au serveur afin d'ajouter le pays dans la base de données.
               \item Une confirmation est envoyée par le serveur.
               \item L'utilisateur est informé que tout s'est bien passé.
            \end{enumerate}     \\
      \hline
      Scénario alternatif&
            \begin{itemize}
               \item[3b1] Une erreur s'est produite sur le serveur.
               \item[3b2] L'action est annulée et l'utilisateur est informé.  
            \end{itemize}\\
      \hline
      Trigger&L'utilisateur appuie sur le bouton permettant d'ajouter un pays\\
      \hline
      Fréquence d'utilisation&Assez souvent.\\
      \hline
   \end{tabular}
\end{table}

\newpage

\subsubsection{Modifier frais pays}
\begin{table}[h]
   \begin{tabular}{|c|p{10cm}|}
      \hline
      Acteur principal&Institution financière\\
      \hline
      Description&Permet à un institution de modifier les frais d'un pays\\
      \hline
      Préconditions&Être connecté à un compte d'une application fiancière.\\
      \hline
      PostConditions&Les frais ont été modifiés.\\
      \hline
      Scénario principal& 
            \begin{enumerate}
               \item L'utilisateur affiche la liste des pays déjà présent dans son institution et choisi le pays à modifier.
               \item L'utilisateur entre le montant des nouveaux frais.
               \item Une demande de modification est envoyée au serveur.
               \item Le serveur renvoie un message disant que tout s'est bien passé.
               \item L'utilisateur est notifié.
            \end{enumerate}     \\
      \hline
      Scénario alternatif&
            \begin{itemize}
               \item[4b1] Une erreur s'est produite.
               \item[4b2] L'utilisateur est notifié qu'une erreur s'est produite et l'action est annulée.  
            \end{itemize}\\
      \hline
      Trigger&L'utilisateur clique sur le bouton permettant de modifier les frais d'un pays.\\
      \hline
      Fréquence d'utilisation&Moyenne\\
      \hline
   \end{tabular}
\end{table}

\section{Extension 5 - VION François}
\subsection{Partie client}
\subsubsection{Voir la liste des assurances}
\begin{table}[h]
    \begin{tabular}{|c|p{10cm}|}
       \hline
       Acteur principal&Le client\\
       \hline
       Description&Le client accède à la liste de ses assurances chez une ou plusieurs banques.\\
       \hline
       Préconditions&Le client est connecté à un compte.\\
       \hline
       PostConditions&La liste des assurances est affichée.\\
       \hline
       Scénario principal& 
             \begin{enumerate}
                \item Une requête est envoyée au serveur afin d'avoir accès à la liste des assurances d'un client.
                \item Le serveur renvoie la liste des assurances d'un client.
                \item La liste des assurances est affichée.
             \end{enumerate}     \\
       \hline
       Scénario alternatif&
        \begin{itemize}
            \item[1a.] Si le client ne possède pas d'assurance, une liste vide est affichée.
            \item[2a.] Une erreur est survenue lors du traitement de la demande par le serveur.
            \item[2b.] Un message d'erreur est affiché à l'écran de l'utilisateur. 
        \end{itemize}     \\
       \\
       \hline
       Trigger&Le client accède à la fenêtre de la liste des assurances.\\
       \hline
       Fréquence d'utilisation&Très souvent\\
       \hline
    \end{tabular}
 \end{table}

 %newpage entre chaque tableau
\newpage

%Nom de votre UseCase
\subsubsection{Voir l'historique des assurances}
\begin{table}[h]
    \begin{tabular}{|c|p{10cm}|}
       \hline
       Acteur principal&Le client\\
       \hline
       Description&Le client accède à l'historique de ses assurances.\\
       \hline
       Préconditions&Le client est connecté à un compte.\\
       \hline
       PostConditions&L'historique des assurances est affiché.\\
       \hline
       Scénario principal& 
             \begin{enumerate}
                \item Une requête est envoyée au serveur afin d'avoir accès à l'historique des assurances d'un client.
                \item Le serveur renvoei l'historique des assurances d'un client.
                \item L'historique des assurances est affiché.
             \end{enumerate}     \\
       \hline
       Scénario alternatif&
        \begin{itemize}
            \item[1a.] Si le client n'a jamais possédé d'assurance, un historique vide est affiché.
            \item[2a.] Une erreur est survenue lors du traitement de la demande par le serveur.
            \item[2b.] Un message d'erreur est affiché à l'écran de l'utilisateur. 
        \end{itemize}
       \\
       \hline
       Trigger&Le client accède à la fenêtre de l'historique des assurances\\
       \hline
       Fréquence d'utilisation&Rare\\
       \hline
    \end{tabular}
 \end{table}

 %newpage entre chaque tableau
\newpage

%Nom de votre UseCase
\subsubsection{Introduire une demande ou un devis}
\begin{table}[h]
    \begin{tabular}{|c|p{10cm}|}
       \hline
       Acteur principal&Le client\\
       \hline
       Description&Le client introduit une demande ou un devis qui sera envoyé à la banque.\\
       \hline
       Préconditions&Le client est connecté à un compte.\\
       \hline
       PostConditions&Le client a introduit une demande ou un devis.\\
       \hline
       Scénario principal& 
             \begin{enumerate}
                \item Une demande ou un devis est créé.
                \item La demande ou le devis est envoyé au serveur.
             \end{enumerate}     \\
       \hline
       Scénario alternatif& 
       \begin{itemize}
        \item[1a.] Une erreur est survenue lors du traitement de la demande par le serveur.
        \item[1b.] Un message d'erreur est affiché à l'écran de l'utilisateur. 
    \end{itemize}
       \\
       \hline
       Trigger&Le client clique sur le bouton d'envoi de la demande ou du devis.\\
       \hline
       Fréquence d'utilisation&Rare\\
       \hline
    \end{tabular}
 \end{table}

 %newpage entre chaque tableau
\newpage

%Nom de votre UseCase
\subsubsection{Annuler une assurance}
\begin{table}[h]
    \begin{tabular}{|c|p{10cm}|}
       \hline
       Acteur principal&Le client\\
       \hline
       Description&Le client annule une de ses assurances.\\
       \hline
       Préconditions&Le client est connecté à un compte et possède au moins une assurance.\\
       \hline
       PostConditions&L'assurance est annulée.\\
       \hline
       Scénario principal& 
             \begin{enumerate}
                \item Le client choisit une assurance a annuler.
                \item Une requête est envoyée au serveur pour annuler l'assurance choisie.
                \item Le serveur renvoie une confirmation et le client est prévenu que l'opération s'est bien effectuée.
             \end{enumerate}     \\
             \subsectionn et le client est prévenu que l'opération s'est bien effectuée.
             \end{enumerate}     \\
       \hline
       Scénario alternatif&    
       \begin{itemize}
        \item[1a.] Une erreur est survenue lors du traitement de la demande par le serveur.
        \item[1b.] Un message d'erreur est affiché à l'écran de l'utilisateur. 
        \item[2a.] Si le code PIN est mauvais 3 fois, le compte est bloqué.
        \end{itemize}
       \\
       \hline
       Trigger&Le client clique sur le bouton de payement.\\
       \hline
       Fréquence d'utilisation&Souvent\\
       \hline
    \end{tabular}
 \end{table}



   %newpage entre chaque tableau
\newpage

\subsection{Banque}
%Nom de votre UseCase
\subsubsection{Envoyer un devis}
\begin{table}[h]
    \begin{tabular}{|c|p{10cm}|}
       \hline
       Acteur principal&L'institution financière\\
       \hline
       Description&L'institution financière envoie un devis suite à une demande du client.\\
       \hline
       Préconditions&L'institution financière est connectée à un compte et a reçu une demande de devis.\\
       \hline
       PostConditions&Un devis est envoyé au client\\
       \hline
       Scénario principal& 
             \begin{enumerate}
                \item L'institution financière crée un devis.
                \item Une requête est envoyée au serveur afin d'envoyer le devis au client.
                \item Le serveur renvoie une confirmation et l'institution financière est prévenue que l'opération s'est bien effectuée.
             \end{enumerate}     \\
       \hline
       Scénario alternatif&    
       \begin{itemize}
        \item[1a.] Une erreur est survenue lors du traitement de la demande par le serveur.
        \item[1b.] Un message d'erreur est affiché à l'écran de l'utilisateur. 
        \end{itemize}
       \\
       \hline
       Trigger&L'institution financière clique sur la fenêtre pour envoyer un devis.\\
       \hline
       Fréquence d'utilisation&Très souvent\\
       \hline
    \end{tabular}
 \end{table}

\end{document}

%---------Template pour la description semi-structurée----------
\begin{itemize}
   \item \textbf{Acteurs:}
   \item \textbf{Description:}
   \item \textbf{Préconditions:}
   \item \textbf{Postconditions:}
   \item \textbf{Scénario principal:}
   \item \textbf{Scénario alternatif:}
   \item \textbf{Trigger:}
   \item \textbf{Fréquence d'utilisation:}
\end{itemize}

\newpage

\subsection{}
\begin{table}[h]
   \begin{tabular}{|c|p{10cm}|}
      \hline
      Acteur principal&Institution fiancière\\
      \hline
      Description&    \\
      \hline
      Préconditions&      \\
      \hline
      PostConditions&      \\
      \hline
      Scénario principal& 
            \begin{enumerate}
               \item test1
               \item test2
               \item test3
            \end{enumerate}     \\
      \hline
      Scénario alternatif&      \\
      \hline
      Trigger&      \\
      \hline
      Fréquence d'utilisation&      \\
      \hline
   \end{tabular}
\end{table}
