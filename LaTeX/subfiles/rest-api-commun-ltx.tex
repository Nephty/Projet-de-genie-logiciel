\documentclass{article}
\usepackage[utf8]{inputenc}
\usepackage[T1]{fontenc}
\usepackage[french]{babel}
\usepackage{amsmath,amsfonts,amssymb,amsthm}
\usepackage[margin=2.5cm]{geometry}

\begin{document}

\paragraph{Introduction}

Ce diagramme décrit le fonctionnement de l'api. Nous l'avons réalisé en se basant
sur le use case diagramme ainsi que celui de relation-overview. On s'est permis quelques
simplifications comme lorsqu'une réponse n'est constitué que d'un code elle est considérée 
comme triviale et n'est donc pas représenté. En revanche si le corps de la réponse est plus
complet (notamment pour les méthodes de login) il sera précisé.
\newline
\newline
4 méthodes HTTP sont utilisées dans ce schéma:
\begin{enumerate}
    \item \textbf{GET}: Méthode utilisée lorsqu'il est nécessaire d'accéder à des données de la BDD.
                        Elle prendra en général comme paramètre(s) le(s) clé(s) primaire de la table
                        ou la recherche est effectuée.
    \item \textbf{POST}: Méthode utilisée lorsqu'il est nécessaire de rajouter des données dans la BDD.
                        Elle requiert en général les attributs de la table ou la création se fera.
    \item \textbf{PUT}: Méthode utilisée lorsqu'il est nécessaire de modifier des données dans la BDD.
                        Elle prend en paramètre le(s) clé(s) primaire de la table ou la modification est
                        effectuée ainsi que le(s) paramètre(s) à modifier. Si ils sont plusieurs, ils seront
                        généralement optionnel et seulement ceux précisé seront modifiés.
    \item \textbf{DELETE}: Méthode utilisée lorsqu'il est nécessaire de supprimer des données dans la BDD.
                            Elle prendra en général comme paramètre(s) le(s) clé(s) primaire de la table 
                            ou la suppression est effectuée.
\end{enumerate}

\paragraph{\textbackslash user}
    \begin{enumerate}
        \item \textbf{GET}: Elle retournera une instance de User. Le paramètre user-id est requis.
        \item \textbf{POST}: Cette méthode est employée lors de la création d'un nouveau compte.
        \item \textbf{PUT}: Elle permet de modifier la langue de préférence de l'utilisateur ou le mot de passe
                            La méthode s'adaptera en fonction des données fournies. Soit le paramètre language aura été spécifié
                            soit les paramètres oldPassword et newPassword.
        \item \textbf{DELETE}: Le paramètre user-id est requis cette méthode supprime simplement un user.
    \end{enumerate}
    \subparagraph{\textbackslash user\textbackslash login}
        \begin{enumerate}
            \item \textbf{GET}: Lorsqu'un user se connecte si les crédentiels entrés sont correct, elle renvoie
                                le profil sinon elle renvoie des informations sur ce qui s'est mal déroulé.
        \end{enumerate}


\paragraph{\textbackslash account}
    \begin{enumerate}
        \item \textbf{GET}: Renvoie une instance d'un account ainsi qu'une liste de tous ses sub-accounts
        \item \textbf{POST}: Utile pour la création d'un nouveau compte. Elle créera notamment un sub-account
                                avec la currency par défaut de la banque associée.
        \item \textbf{DELETE}: Suppression d'un compte ainsi que les sub-account et account-access associés
    \end{enumerate}
    \subparagraph{\textbackslash account\textbackslash account-access}
        \begin{enumerate}
            \item \textbf{POST}: Création d'un nouvel accès.
            \item \textbf{PUT}: Modifier le statut d'un accès le paramètre write ou hidden sera spécifié. accountID
                                et userID correspondent aux clés primaires de la table.
            \item \textbf{DELETE}: account-id et user-id correspondent aux clés primaires de la table.
        \end{enumerate}

    \subparagraph{\textbackslash account\textbackslash sub-account}
        \begin{enumerate}
            \item \textbf{POST}: Crée un nouveau sub-account
            \item \textbf{PUT}: Modifie la balance dans un account. Le paramètre amount reprrésente ce qu'il faut
                                rajouter, il peut être négatif.
            \item \textbf{DELETE}: Suppression d'un sub-account. Les paramètres sont les clés primaires de la table.
        \end{enumerate}


\paragraph{\textbackslash notification}
    \begin{enumerate}
        \item \textbf{GET}: Renvoie la notification correspondant à l'id
        \item \textbf{POST}: Crée une notification. Le paramètre fromBank indique si le destinateur et la banque
                            ou l'utilisateur ce qui sera utile pour savoir dans quelle table rajouter la notification.
        \item \textbf{DELETE}: Suppression de la notification correspondant à l'id
    \end{enumerate}


\paragraph{\textbackslash bank}
    \begin{enumerate}
        \item \textbf{GET}: Retourne une instance de banque.
        \item \textbf{POST}: Création d'une banque.
        \item \textbf{DELETE}: Suppression d'une banque, swift est la clé primaire de la table.
        \item \textbf{PUT}: Les paramètres modifiables sont password, adress, country et defaultCurrency
    \end{enumerate}

    \subparagraph{\textbackslash bank\textbackslash login}
        \begin{enumerate}
            \item \textbf{GET}: Lorsqu'une banque se connecte si les crédentiels entrés sont correct, elle renvoie
                                le profil sinon elle renvoie des informations sur ce qui s'est mal déroulé.
        \end{enumerate}

    \subparagraph{\textbackslash bank\textbackslash customer}
        \begin{enumerate}
            \item \textbf{GET}: Renvoie la liste de tous les clients appartenant à la banque dont le swift eest
                                passé en paramètre.
            \item \textbf{POST}: Ajoute un nouveau client.
            \item \textbf{DELETE}: Supprime le client (customer-id) de la bank (swift). Les paramètres sont les clés primaires.
        \end{enumerate}


\paragraph{\textbackslash transaction}
    \begin{enumerate}
        \item \textbf{GET}: Renvoie une liste des transactions d'un portefeuille qu'il en soit l'émetteur ou le receveur
        \item \textbf{POST}: Crée une nouvelle transaction.
    \end{enumerate}


\end{document}