\section{Introduction}

Ce rapport contient les diagrammes UML ainsi que des diagrammes d'entités relation, un schéma de REST API et des maquettes d'interface utilisateur.
Ces diagrammes modélisent les divers éléments du projet de Génie logiciel - modélisation.\\
Pour rappel, cette étape a pour objectif de développer un ensemble de diagramme qui nous servirons de base pour la phase d'implémentation.
La phase d'implémentation comporte deux application : Une application de gestion de portefeuilles financiers et une application pour une institution financière.\\
De plus, chaque participants avait une extension personnelle qui devait être modélisée individuellement.
Augustin devait modéliser un support pour la gestion des cartes bancaire, 
Cyril devait modéliser une gestion de différentes devises ainsi que la possibilité d'effecuter des virements internationaux,
François devait modéliser une gestion de contrats d'assurance,
et Arnaud devait modéliser des paiements par QR code, les paiements avec et sans contacts ainsi que la gestion de fraude.

\section{Répartition des tâches}
\begin{itemize}
    \item Diagrammes de cas d'utilisation : Cyril \textsc{Moreau}
    \item Diagrammes d'interaction : Arnaud \textsc{Moreau}
    \item Diagrammes de classes : François \textsc{Vion}
    \item Diagrammes de séquences : François \textsc{Vion} \& Augustin \textsc{Houba}
    \item Modèle de données : Cyril \textsc{Moreau}
    \item Design du REST API : Augustin \textsc{Houba}
    \item Maquettes des interfaces : Arnaud \textsc{Moreau}
\end{itemize}

\newpage

\section{Outils utilisés}

\subsection{GitHub} La célèbre plateforme GitHub permet à différentes personnes de collaborer et ajouter leurs modifications à un ensemble de fichiers, généralement du code. 
En vue de la deuxième phase d'implémentation et afin de faciliter la mise en place du rapport, un dépôt GitHub a été créé. 
Ce dernier a servi à écrire le rapport de modélisation et servira à l'implémentation de ce projet.

\subsection{Trello} Trello est un outil de coordination et de gestion d'équipe. 
Il permet d'assigner des tâches à des participants, créer des "TO-DO lists", créer des checklists, 
poser des deadlines, et bien plus encore. Sa simplicité d'utilisation, ses possibilités d'automatisation et son design visuel très intuitif et 
user-friendly fait de Trello un outil de productivité très puissant permettant d'économiser beaucoup de temps dépensé à l'organisation et les répartition des tâches.

\subsection{Discord} Le réseau social Discord a été utilisé afin de communiquer entre nous et se partager des fichiers.
En effet, ce réseau est bien adapté à ce genre d'utilisation grâce à son organisation en différent canaux de communication. 
Ce réseau social est très similaire à Slack mais reste plus pratique à utiliser car nous sommes souvent connectés à celui-ci.
De plus, nous avons une meilleure maîtrise de Discord que de Slack.

\subsection{Draw.io} L'outil Draw.io a servi à la création de premiers diagrammes "brouillons". 
Cet outil permet de créer des diagrammes UML. 
L'avantage de cette application est qu'elle est très simple et intuitive d'utilisation et permet de réaliser des petits diagrammes UML,
mais le désavantage est qu'elle ne contient pas la même quantité de fonctionnalités que Visual Paradigm. 
Ce compromis est idéal afin de créer des premières esquisses.

\subsection{Visual Paradigm} L'outil Visual Paradigm a été utilisé afin de réalisé les diagrammes UML. 
Chaque étudiant possédait son projet (fichier au format .vpp) contenant ses diagrammes UML correspondant à son extension, 
et un projet (fichier au format .vpp) commun contenait les diagrammes correspondants à la partie commune.

\subsection{Yakindu Statechart Tools} L'outil Yakindu Statechart Tools a permis de réaliser le diagramme d'états comportementaux.
\newpage