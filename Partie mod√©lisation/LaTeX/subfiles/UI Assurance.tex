\paragraph{Introduction}
\noindent
Cete section décrit l'utilisation de l'interface graphique de l'extension assurance en se basant sur la maquette de l'interface. Seule les deux dernières fenêtres sont utilisées dans l'application pour banque, toutes les autres font partie de l'application client. Par soucis de clarte, ce diagramme est disponible en annexe du rapport. \\

\paragraph{Fenêtres disponibles}
\begin{itemize}
\item Main screen
\item Insurance main screen
\item Informations on the insurances
\item Submit a quote
\item Received quotation
\item Insurance manager
\item Insurance history
\item Register for insurances
\item Car insurance registration
\item Life insurance payment
\item Damage insurance payment
\item Pension savings insurance payment
\item Payment
\item Enter PIN to confirm
\item Life insurance manager
\item Withdraw
\item Damage insurance manager
\item Pension savings insurance manager
\item Requests
\item Quotations
\end{itemize}


\paragraph{Fenêtre \emph{Main screen}}
\access{en se connectant à l'application via la fenêtre Sign in, en ayant terminé un payement, en appuyant sur le bouton back de la fenêtre Insurance main screen ou via les fenêtre Change language ou Change password.}
\content{les boutons Sign out, Language, Insurance, Financial product, Requests et Change password.}
\begin{itemize}
\item \navbutton{Sign out}{Sign out} ;
\item \navbutton{Language}{Change language} ;
\item \navbutton{Insurance}{Insurance main screen}.
\item \navbutton{Financial product}{Financial product}
\item \navbutton{Requests}{Requests}
\item \navbutton{Change password}{Change password}
\end{itemize}


\paragraph{Fenêtre \emph{Insurance main screen}}
\access{en cliquant sur Insurance dans main screen, en ayant annulée une assurance, ou en cliquant sur le bouton des back des fenêtres Sumit a quote, Informations on the insurances et Insurance manager.}
\content{Les boutons Back, Insurance manager, Submit a quote, Informations on the insurances}
\begin{itemize}
\item \navbutton{Back}{Main screen} ;
\item \navbutton{Insurance manager}{Insurance manager}
\item \navbutton{Submit a quote}{Submit a quote} ;
\item \navbutton{Informations on the insurances}{Informations on the insurances}
\end{itemize}


\paragraph{Fenêtre \emph{Informations on the insurances}}
\access{en cliquant sur le bouton Informations on the insurances de la fenêtre Insurance main screen}
\content{Le bouton back et des zones de texte}
\begin{itemize}
\item \navbutton{Back}{Insurance main screen} ;
\item Les zones de textes contiennent des informations sur les assurances
\end{itemize}


\paragraph{Fenêtre \emph{Submit a quote}}
\access{en cliquant sur le bouton Submit a quote de la fenêtre Insurance main screenou via le bouton Back de la fenêtre Received quotation}
\content{Les boutons Back, Received quotation et Submit a quote ainsi que les listes de sélection Choose a bank et Choose an insurance.}
\begin{itemize}
\item \navbutton{Back}{Insurance main screen} ;
\item \navbutton{Reveived quotation}{Received quotation}
\item Le bouton Submit a quote qui effectue la demande de devis.
\item La liste sélectionnable Choose a bank.
\item La liste sélectionnable Choose an insurance.
\end{itemize}


\paragraph{Fenêtre \emph{Received quotation}}
\access{en cliquant sur le bouton Received quotation de la fenêtre Submit a quote}
\content{Les boutons Back et Register for insurance et la liste des devis reçus.}
\begin{itemize}
\item \navbutton{Back}{Submit a quote} ;
\item Le bouton Register for insurance qui envoie l'utilisateur vers la fenêtre Life insurance payment s'il a choisi une assurance vie, vers la fenêtre Damage insurance payment s'il a choisi une assurance famille, assistance, maison ou hopital, vers la fenêtre Car insurance registration s'il a choisi une assurance voiture et vers la fenêtre Pension savings insurance payment s'il a choisi une assurance épargne pension.
\item La liste sélectionnable des devis reçus
\end{itemize}


\paragraph{Fenêtre \emph{Insurance manager}}
\access{en cliquant sur le bouton Insurance manager de la fenêtre Insurance main screen ou en cliquand sur les boutons Back des fenêtres Insurance history, Life insurance manager, Pension savings insurance manager et Damage insurance manager.}
\content{Les boutons Back, Register for insurance, Pay the annual premium, See insurance's history, Manage insurance et Cancel insurance ainsi que la liste sélectionnable Choose an insurance.}
\begin{itemize}
\item \navbutton{Back}{Insurance main screen} ;
\item \navbutton{Register for insurance}{Register for insurance}
\item Le bouton Pay the annual premium qui envoie l'utilisateur vers la fenêtre Life insurance payment s'il a choisi une assurance vie, vers la fenêtre Damage insurance payment s'il a choisi une assurance famille, assistance, maison ou hopital, vers la fenêtre Car insurance registration s'il a choisi une assurance voiture et vers la fenêtre Pension savings insurance payment s'il a choisi une assurance épargne pension.
\item \navbutton{See insurance's history}{Insurance history}
\item Le bouton Manage insurace qui envoie l'utilisateur vers la fenêtre Life insurance manager s'il a choisi une assurance vie, vers la fenêtre Damage insurance manager s'il s'agit d'une assurance de dégâts ou vers la fenêtre Pension savings insurance manager s'il s'agit d'une assurance épargne pension.
\item Le bouton Cancel insurance qui annule l'assurance selectionnée.
\item La liste sélectionnable des assurances de l'utilisateur.
\end{itemize}


\paragraph{Fenêtre \emph{Insurance history}}
\access{en cliquant sur le bouton See insurance's history de la fenêtre Insurance manager.}
\content{Un bouton Back et la liste de l'historique des assurances.}
\begin{itemize}
\item \navbutton{Back}{Insurance manager} ;
\item La liste de l'historique des assurances avec les assurances terminée surlignées en gris.
\end{itemize}


\paragraph{Fenêtre \emph{Register for insurances}}
\access{en cliquant sur le bouton Register for insurance de la fenêtre Insurance manager ou Received quotation ou en cliquand sur le bouton Back des fenêtres Car insurance registration, Life insurance payment, Damage insurance payment ou Pension savings payment}
\content{Les boutons Back et Registration ainsi que les listes sélectionnables Choose a bank et Choose an insurance.}
\begin{itemize}
\item Le bouton Back qui renvoie l'utilisateur à la fenêtre précédente.
\item La liste sélectionnable Choose a bank.
\item La liste sélectionnable Choose an insurance.
\item Le bouton Registration qui envoie l'utilisateur vers la fenêtre Life insurance payment s'il a choisi une assurance vie, vers la fenêtre Damage insurance payment s'il a choisi une assurance famille, assistance, maison ou hopital, vers la fenêtre Car insurance registration s'il a choisi une assurance voiture et vers la fenêtre Pension savings insurance payment s'il a choisi une assurance épargne pension.
\end{itemize}


\paragraph{Fenêtre \emph{Car insurance registration}}
\access{en cliquant sur le bouton Registration de la fenêtre Register for insurance ou le bouton Register for insurance de la fenêtre Received quotation après voir selectionné une assurance voiture.}
\content{Les boutons Back et Pay the annual prime, les informations sur l'assurance choisie ainsi que des checkBox pour les options.}
\begin{itemize}
\item \navbutton{Back}{Register for insurance} ;
\item Les informations sur le type d'assurance
\item Les checkBox Omnium insurance, Driver insurance et Legal protection Insurance
\item \navbutton{Pay the annual prime}{Payment}
\end{itemize}


\paragraph{Fenêtre \emph{Life insurance payment}}
\access{en cliquant sur le bouton Registration de la fenêtre Register for insurance ou le bouton Register for insurance de la fenêtre Received quotation après voir selectionné une assurance vie.}
\content{Les boutons Back et Add funds ainsi que les informations sur l'assurance.}
\begin{itemize}
\item Le bouton renvoie à la fenêtre précedent, soit Register for insurance, soit Received quotation. ;
\item Les informations sur l'assurance
\item \navbutton{Add funds}{Payment}
\end{itemize}



\paragraph{Fenêtre \emph{Damage insurance payment}}
\access{en cliquant sur le bouton Registration de la fenêtre Register for insurance ou le bouton Register for insurance de la fenêtre Received quotation après voir selectionné une assurance famille, assistance, habitation ou hopital.}
\content{Ls boutons Back et Pay the annual prime ainsi que les informations sur l'assurance.}
\begin{itemize}
    \item Le bouton renvoie à la fenêtre précedent, soit Register for insurance, soit Received quotation. ;
    \item Les informations sur l'assurance
    \item \navbutton{Pay the annual prime}{Payment}
\end{itemize}

\paragraph{Fenêtre \emph{Pension savings insurance payment}}
\access{en cliquant sur le bouton Registration de la fenêtre Register for insurance ou le bouton Register for insurance de la fenêtre Received quotation après voir selectionné une assurance épargne pension.}
\content{Ls boutons Back et Pay the annual prime ainsi que les informations sur l'assurance.}
\begin{itemize}
    \item Le bouton renvoie à la fenêtre précedent, soit Register for insurance, soit Received quotation. ;
    \item Les informations sur l'assurance
    \item \navbutton{Pay the annual prime}{Payment}
\end{itemize}

\paragraph{Fenêtre \emph{Payment}}
\access{en cliquant sur le bouton Pay the annual prim ou add funds des fenêtres Life insurance payment, Damage insurance payment, Pension savings insurance payment, Pension savings insurance manager ou Life insurance manager.}
\content{Les boutons Back et Confirm, les textFields Amount, IBAN, Message et Date qui sont tous vérouillées sauf les cas où il s'agit d'un montant libre,alors celui-ci est dévérouiller. Une liste déroulante Account.}
\begin{itemize}
\item Le bouton Back renvoie à la fenêtre précédente.
\item \textfield{Amount}{le montant libre}
\item La liste déroulante Account qui permet de choisir le compte débité.
\item \textfield{IBAN}{le compte IBAN du receveur}
\item \textfield{Message}{une communication pour la transaction}
\item \textfield{Date}{la date souhaitée pour la transaction}
\end{itemize}


\paragraph{Fenêtre \emph{Enter PIN to confirm}}
\access{en cliquant sur le bouton Confirm de la fenêtre Payment ou Withdraw de la fenêtre Withdraw.}
\content{Les boutons Back et Confirm. Un textField pour rentrer son PIN ainsi que 10 boutons numérotés pour le taper à la souris.}
\begin{itemize}
\item \navbutton{Back}{Payment} ;
\item \textfield{PIN}{son code PIN}
\item Les 10 boutons numérotés qui entrent les numéros dans le textField PIN
\item Le bouton Confirm qui effectue la transaction si le code PIN est correct, si c'est le cas, l'utilisateur est redirigé vers la fenêtre Main screen.
\end{itemize}


\paragraph{Fenêtre \emph{Life insurance manager}}
\access{en cliquant sur le bouton Manage insurance de la fenêtre Insurance manager en ayant selectionné une assurance vie.}
\content{Les boutons Back, Add funds et Withdraw funds ainsi que des informations sur l'assurance.}
\begin{itemize}
\item \navbutton{Back}{Insurance manager} ;
\item \navbutton{Add funds}{Payment}
\item \navbutton{Withdraw funds}{Withdraw}
\item Les informations de l'assurance.
\end{itemize}


\paragraph{Fenêtre \emph{Withdraw}}
\access{en cliquant sur le bouton Withdraw funds de la fenêtre Life insurance manager.}
\content{Les boutons Back et Withdraw, le textField amount et la liste déroulante Receiver account.}
\begin{itemize}
\item \navbutton{Back}{Life insurance manager} ;
\item \navbutton{Withdraw}{Enter PIN to confirm}
\item \textfield{Amount}{le montant a retirer}
\item La liste déroulante pour choisir le compte receveur.
\end{itemize}


\paragraph{Fenêtre \emph{Damage insurance manager}}
\access{en cliquant sur le bouton Manage insurance de la fenêtre Insurance manager en ayant selectionné une assurance de dégâts.}
\content{Les boutons Back et Make changes, les informations sur l'assurance choisie et les checkBox Automatic renewal et Omnium insurance, Driver insurance et Legal protection insurance dans le cas d'une assurance voiture.}
\begin{itemize}
\item \navbutton{Back}{Insurance manager} ;
\item Le bouton Make change qui effectue les changements
\item La checkBox Automatic renewal a cocher/décocher
\item Les checkBox Omnium insurance, Driver insurance et Legal protection insurance a cocher/décocher dans le cas d'une assurance voiture.
\end{itemize}


\paragraph{Fenêtre \emph{Pension savings insurance manager}}
\access{en cliquant sur le bouton Manage insurance de la fenêtre Insurance manager en ayant selectionné une assurance épargne pension.}
\content{Les boutons Back, Add funds ainsi que des informations sur l'assurance.}
\begin{itemize}
\item \navbutton{Back}{Insurance manager} ;
\item \navbutton{Add funds}{Payment}
\item Les informations de l'assurance.
\end{itemize}


\paragraph{Fenêtre \emph{Requests}}
\access{en cliquant sur le bouton Requests de la fenêtre main screen}
\content{Les boutons Back, Quotations, Transfer permission requests et Portfolio requests}
\begin{itemize}
\item \navbutton{Back}{Main screen} ;
\item \navbutton{Quotations}{Quotations}
\item \navbutton{Transfer permission requests}{Transfer permission requests}
\item \navbutton{Portfolio requests}{Portfolio requests}
\end{itemize}

\paragraph{Fenêtre \emph{Quotations}}
\access{en cliquant sur le bouton Quotations de la fenêtre Requests}
\content{Les boutons Back et Send quotation ainsi que la liste sélectionnable des demandes de devis}
\begin{itemize}
\item \navbutton{Back}{Requests} ;
\item Le bouton Send quotation qui envoie le devis au client.
\item La liste sélectionnable des demande de devis reçues.
\end{itemize}