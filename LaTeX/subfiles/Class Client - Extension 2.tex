\paragraph{Introduction} Contrairment à la partie commune, les diagrammes de classes des deux applications sont légèrement différent.
Certaines classes ne se trouve uniquement dans l'une ou l'autre application.

\paragraph{Partie logique}
Tout d'abord, un changement majeur a été effectué dans le gestion des comptes de la partie commune à cause de cette extension.
En effect, dans la base de donnée, un compte est divisé en sous-compte en fonction de leur devise. 
Dans la partie commune, uniquement l'EUR est implémenté donc il n'ont qu'un seul sous compte.\\
Cela implique donc quelques changement dans les classes correspondant aux comptes : \\

\begin{itemize}
    \item Une classe \emph{SubAccount} a été ajoutée à cause de ce changement dans la base de donnée (Dans le cas de cette extension, cela permet de gérer les comptes multidevise).
    \item La classe \emph{Account} contient maintenant une liste de \emph{SubAccount}.
    \item Une enumeration \emph{SEPACountries} a été ajoutée afin de garder la liste des pays de la zone SEPA. Cette liste n'est pas modifiée très souvent, une enumeration est donc suffisante.
    \item Une enumeration \emph{Currencies} a également été ajoutée afin de garder la liste des devises supportées par l'application.
        Lors de l'implémentation, cette technique pourrait être modifiée afin de permettre aux institution d'ajouter de nouvelles devises.
    \item Une nouvelle classe \emph{ExchangeRateViewer} a été ajoutée afin de permettre à l'application de connaitre les taux de change entre 2 devises et de convertir un montant d'une devise à une autre.
    \item La classe \emph{Transaction} a également été adaptée afin de permettre d'effectuer des virements internationnaux et des virements entre 2 comptes ayant des devises différentes.
    \item La classe \emph{Address} permet de regrouper les données de l'address du bénéficaire d'une transaction (city, street,...).
\end{itemize}

\paragraph{Partie API}
Ici, très peu de changement ont été effectué par rapport à la partie commune.
Uniquement 2 controller ont été ajouté afin d'accéder/modifier des information sur le serveur.

\paragraph{Partie GUI}
Finalementn, uniquement 2 nouvelles classes correspondant aux 2 nouvelles fenêtres ont été ajoutée. Toujours en respectant le design pattern singleton.