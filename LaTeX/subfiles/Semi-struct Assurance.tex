\documentclass[]{article}

\usepackage[utf8]{inputenc}
\usepackage[T1]{fontenc}
\usepackage[frenchb]{babel}
\usepackage{amsmath,amsfonts,amssymb,amsthm}
\usepackage{graphicx}

\begin{document}

\section{Extension assurance}

%Nom de votre UseCase
\subsection{Voir la liste des assurances}
\begin{table}[h]
    \begin{tabular}{|c|p{10cm}|}
       \hline
       Acteur principal&Le client\\
       \hline
       Description&Le client accède à la liste de ses assurances chez une ou plusieurs banques.\\
       \hline
       Préconditions&Le client est connecté à un compte.\\
       \hline
       PostConditions&La liste des assurances est affichée.\\
       \hline
       Scénario principal& 
             \begin{enumerate}
                \item Une requête est envoyée au serveur afin d'avoir accès à la liste des assurances d'un client.
                \item Le serveur renvoie la liste des assurances d'un client.
                \item La liste des assurances est affichée.
             \end{enumerate}     \\
       \hline
       Scénario alternatif&
        \begin{itemize}
            \item[1a.] Si le client ne possède pas d'assurance, une liste vide est affichée.
            \item[2a.] Une erreur est survenue lors du traitement de la demande par le serveur.
            \item[2b.] Un message d'erreur est affiché à l'écran de l'utilisateur. 
        \end{itemize}     \\
       \\
       \hline
       Trigger&Le client accède à la fenêtre de la liste des assurances.\\
       \hline
       Fréquence d'utilisation&Très souvent\\
       \hline
    \end{tabular}
 \end{table}

 %newpage entre chaque tableau
\newpage

%Nom de votre UseCase
\subsection{Voir l'historique des assurances}
\begin{table}[h]
    \begin{tabular}{|c|p{10cm}|}
       \hline
       Acteur principal&Le client\\
       \hline
       Description&Le client accède à l'historique de ses assurances.\\
       \hline
       Préconditions&Le client est connecté à un compte.\\
       \hline
       PostConditions&L'historique des assurances est affiché.\\
       \hline
       Scénario principal& 
             \begin{enumerate}
                \item Une requête est envoyée au serveur afin d'avoir accès à l'historique des assurances d'un client.
                \item Le serveur renvoei l'historique des assurances d'un client.
                \item L'historique des assurances est affiché.
             \end{enumerate}     \\
       \hline
       Scénario alternatif&
        \begin{itemize}
            \item[1a.] Si le client n'a jamais possédé d'assurance, un historique vide est affiché.
            \item[2a.] Une erreur est survenue lors du traitement de la demande par le serveur.
            \item[2b.] Un message d'erreur est affiché à l'écran de l'utilisateur. 
        \end{itemize}
       \\
       \hline
       Trigger&Le client accède à la fenêtre de l'historique des assurances\\
       \hline
       Fréquence d'utilisation&Rare\\
       \hline
    \end{tabular}
 \end{table}

 %newpage entre chaque tableau
\newpage

%Nom de votre UseCase
\subsection{Introduire une demande ou un devis}
\begin{table}[h]
    \begin{tabular}{|c|p{10cm}|}
       \hline
       Acteur principal&Le client\\
       \hline
       Description&Le client introduit une demande ou un devis qui sera envoyé à la banque.\\
       \hline
       Préconditions&Le client est connecté à un compte.\\
       \hline
       PostConditions&Le client a introduit une demande ou un devis.\\
       \hline
       Scénario principal& 
             \begin{enumerate}
                \item Une demande ou un devis est créé.
                \item La demande ou le devis est envoyé au serveur.
             \end{enumerate}     \\
       \hline
       Scénario alternatif& 
       \begin{itemize}
        \item[1a.] Une erreur est survenue lors du traitement de la demande par le serveur.
        \item[1b.] Un message d'erreur est affiché à l'écran de l'utilisateur. 
    \end{itemize}
       \\
       \hline
       Trigger&Le client clique sur le bouton d'envoi de la demande ou du devis.\\
       \hline
       Fréquence d'utilisation&Rare\\
       \hline
    \end{tabular}
 \end{table}

 %newpage entre chaque tableau
\newpage

%Nom de votre UseCase
\subsection{Annuler une assurance}
\begin{table}[h]
    \begin{tabular}{|c|p{10cm}|}
       \hline
       Acteur principal&Le client\\
       \hline
       Description&Le client annule une de ses assurances.\\
       \hline
       Préconditions&Le client est connecté à un compte et possède au moins une assurance.\\
       \hline
       PostConditions&L'assurance est annulée.\\
       \hline
       Scénario principal& 
             \begin{enumerate}
                \item Le client choisit une assurance a annuler.
                \item Une requête est envoyée au serveur pour annuler l'assurance choisie.
                \item Le serveur renvoie une confirmation et le client est prévenu que l'opération s'est bien effectuée.
             \end{enumerate}     \\
       \hline
       Scénario alternatif&  
       \begin{itemize}
        \item[1a.] Une erreur est survenue lors du traitement de la demande par le serveur.
        \item[1b.] Un message d'erreur est affiché à l'écran de l'utilisateur. 
        \end{itemize}    \\
       \hline
       Trigger&Le client clique sur le bouton d'annulation d'assurance.\\
       \hline
       Fréquence d'utilisation&Très rare\\
       \hline
    \end{tabular}
 \end{table}

 %newpage entre chaque tableau
\newpage

%Nom de votre UseCase
\subsection{Gérer le montant sur une assurance}
\begin{table}[h]
    \begin{tabular}{|c|p{10cm}|}
       \hline
       Acteur principal&Le client\\
       \hline
       Description&Permet d'ajouter et de retirer de l'argent d'une assurance vie.\\
       \hline
       Préconditions&Avoir selectionné une assurance vie et être connecté à un compte.\\
       \hline
       PostConditions&De l'argent a été rajouté ou retiré de l'assurance vie.\\
       \hline
       Scénario principal& 
             \begin{enumerate}
                \item Le client rentre les données nécessaires à l'ajout ou au retrait d'argent.
                \item La transaction est créée et envoyée au serveur
                \item Le serveur renvoie une confirmation et le client est prévenu que l'opération s'est bien effectuée.
             \end{enumerate}     \\
       \hline
       Scénario alternatif&            
       \begin{itemize}
        \item[1a.] Une erreur est survenue lors du traitement de la demande par le serveur.
        \item[1b.] Un message d'erreur est affiché à l'écran de l'utilisateur. 
        \item[2a.] Le client a rentré un mauvais montant et il en est averti.
    \end{itemize}\\
       \hline
       Trigger&Le client clique sur le bouton pour ajouter des fonds ou pour en retirer.\\
       \hline
       Fréquence d'utilisation&Rare\\
       \hline
    \end{tabular}
 \end{table}

 %newpage entre chaque tableau
\newpage

%Nom de votre UseCase
\subsection{Modifier les paramètres d'une assurance}
\begin{table}[h]
    \begin{tabular}{|c|p{10cm}|}
       \hline
       Acteur principal&Le client\\
       \hline
       Description&Le client modifie les paramètres d'une assurance.\\
       \hline
       Préconditions&Le client est connecté à un compte et a selectionné une assurance.\\
       \hline
       PostConditions&Les paramètres d'une assurance ont été modifiés.\\
       \hline
       Scénario principal& 
             \begin{enumerate}
                \item Le client choisit un ou des paramètres a modifier.
                \item Une requête est envoyée au serveur afin de modifier l'assurance.
                \item L'assurance est modifiée
                \item Le serveur renvoie une confirmation et le client est prévenu que l'opération s'est bien effectuée.
             \end{enumerate}     \\
       \hline
       Scénario alternatif&   
       \begin{itemize}
        \item[1a.] Une erreur est survenue lors du traitement de la demande par le serveur.
        \item[1b.] Un message d'erreur est affiché à l'écran de l'utilisateur. 
    \end{itemize}
       \\
       \hline
       Trigger&Le client coche les paramètres a modifier et clique sur le bouton pour appliquer les changements.\\
       \hline
       Fréquence d'utilisation&Rare\\
       \hline
    \end{tabular}
 \end{table}

 %newpage entre chaque tableau
\newpage

%Nom de votre UseCase
\subsection{Obtenir des informations sur les assurances}
\begin{table}[h]
    \begin{tabular}{|c|p{10cm}|}
       \hline
       Acteur principal&Le client\\
       \hline
       Description&Le client reçois des informations sur les assurances.\\
       \hline
       Préconditions&Le client est connecté à un compte.\\
       \hline
       PostConditions&Le client reçois les informations sur les assurances.\\
       \hline
       Scénario principal&Les informations sur les assurances sont affichées.\\
       \hline
       Scénario alternatif&\\
       \hline
       Trigger&Le client accède à la fenêtre des informations sur les assurances.\\
       \hline
       Fréquence d'utilisation&Parfois\\
       \hline
    \end{tabular}
 \end{table}

  %newpage entre chaque tableau
\newpage

%Nom de votre UseCase
\subsection{S'inscrire à une assurance}
\begin{table}[h]
    \begin{tabular}{|c|p{10cm}|}
       \hline
       Acteur principal&Le client\\
       \hline
       Description&Le client s'inscrit à une assurance\\
       \hline
       Préconditions&Le client est connecté à un compte\\
       \hline
       PostConditions&Le client s'est inscrit à un assurance\\
       \hline
       Scénario principal& 
             \begin{enumerate}
                \item Le client choisit une banque et une assurance voulue.
                \item Une requête est envoyée au serveur afin d'avoir les informations sur cette assurance.
                \item Les informations récapitulatives sont affichées au client.
                \item Le client paye la prime annuelle pour confirmer l'inscription.
             \end{enumerate}     \\
       \hline
       Scénario alternatif& 
       \begin{itemize}
        \item[1a.] Une erreur est survenue lors du traitement de la demande par le serveur.
        \item[1b.] Un message d'erreur est affiché à l'écran de l'utilisateur. 
        \item[2a.] Le payement n'a pas été effectué et l'utilisateur en est prévenu.
    \end{itemize}
         \\
       \hline
       Trigger&Cliquer sur le bouton pour s'inscrire à une assurance.\\
       \hline
       Fréquence d'utilisation&Très rare\\
       \hline
    \end{tabular}
 \end{table}


  %newpage entre chaque tableau
\newpage

%Nom de votre UseCase
\subsection{Payer la prime annuelle}
\begin{table}[h]
    \begin{tabular}{|c|p{10cm}|}
       \hline
       Acteur principal&Le client\\
       \hline
       Description&Le client paye la prime annuelle d'une assurance.\\
       \hline
       Préconditions&Le client est connecté à un compte et a selectionné un compte.\\
       \hline
       PostConditions&Le client a payé la prime annuelle d'une assurance\\
       \hline
       Scénario principal& 
             \begin{enumerate}
                \item Le client choisit le compte avec lequel il souhaite payer.
                \item Le client paye en validant son code PIN
                \item Si le code PIN est bon, le serveur renvoie une confirmation et le client est prévenu que l'opération s'est bien effectuée.
             \end{enumerate}     \\
       \hline
       Scénario alternatif&    
       \begin{itemize}
        \item[1a.] Une erreur est survenue lors du traitement de la demande par le serveur.
        \item[1b.] Un message d'erreur est affiché à l'écran de l'utilisateur. 
        \item[2a.] Si le code PIN est mauvais 3 fois, le compte est bloqué.
        \end{itemize}
       \\
       \hline
       Trigger&Le client clique sur le bouton de payement.\\
       \hline
       Fréquence d'utilisation&Souvent\\
       \hline
    \end{tabular}
 \end{table}



   %newpage entre chaque tableau
\newpage
\section{Banque}
%Nom de votre UseCase
\subsection{Envoyer un devis}
\begin{table}[h]
    \begin{tabular}{|c|p{10cm}|}
       \hline
       Acteur principal&L'institution financière\\
       \hline
       Description&L'institution financière envoie un devis suite à une demande du client.\\
       \hline
       Préconditions&L'institution financière est connectée à un compte et a reçu une demande de devis.\\
       \hline
       PostConditions&Un devis est envoyé au client\\
       \hline
       Scénario principal& 
             \begin{enumerate}
                \item L'institution financière crée un devis.
                \item Une requête est envoyée au serveur afin d'envoyer le devis au client.
                \item Le serveur renvoie une confirmation et l'institution financière est prévenue que l'opération s'est bien effectuée.
             \end{enumerate}     \\
       \hline
       Scénario alternatif&    
       \begin{itemize}
        \item[1a.] Une erreur est survenue lors du traitement de la demande par le serveur.
        \item[1b.] Un message d'erreur est affiché à l'écran de l'utilisateur. 
        \end{itemize}
       \\
       \hline
       Trigger&L'institution financière clique sur la fenêtre pour envoyer un devis.\\
       \hline
       Fréquence d'utilisation&Très souvent\\
       \hline
    \end{tabular}
 \end{table}

\end{document}