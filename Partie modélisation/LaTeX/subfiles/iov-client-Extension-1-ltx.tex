\documentclass{article}
\usepackage[utf8]{inputenc}
\usepackage[T1]{fontenc}
\usepackage[french]{babel}
\usepackage{amsmath,amsfonts,amssymb,amsthm}
\usepackage[margin=2.5cm]{geometry}

\begin{document}

\paragraph{Introduction}
Pour mon extension qui est la gestion de cartes je n'ai pas jugé nécessaires de modifier
l'interaction overview diagram de l'application pour l'institution car il n'y avait rien à changé
Je vais donc uniquement parler des ajouts que j'ai réalisé à celui de l'application client.

\paragraph{Gestions de cartes}
Cette partie sert de point d'entrée pour les fonctionnalités de cette extension.
L'utilisateur a plusieurs options disponibles:
\begin{enumerate}
    \item Payer avec une carte
    \item Voir la liste des cartes
\end{enumerate}

\subparagraph{Payer avec une carte}
Ici l'utilisateur pour faire un paiement en utilisant une de ses cartes. Il pourra:
\begin{enumerate}
    \item Annuler le paiement
    \item Confirmer le paiement
\end{enumerate}

Dans les deux cas il sera renvoyer vers la gestion de cartes.

\subparagraph{Voir la liste des cartes}
Cette partie couvre les flows de la gestion de liste des cartes de crédit et de débit
car ils sont similaires. Lorsque l'utilisateur à la liste devant lui, il peut:

\begin{enumerate}
    \item Créer une carte
    \item Sélectionner une carte
\end{enumerate}

\subparagraph{Créer une carte}
Pour cette interaction l'utilisateur aura la possibilité de valider sa demande de création ou
d'annuler. Les deux cas aboutiront à un retour à la liste des cartes.
\subparagraph{Sélectionner une carte}
Lorsque l'utilisateur sélectionne une carte il a la possibilité de:
\begin{enumerate}
    \item Modifier une carte -> Retourne directement à la sélection de carte
    \item Bloquer une carte -> Retourne directement à la sélection de carte
    \item Voir l'historique des transactions de la carte
\end{enumerate}

\subparagraph{Voir l'historique des transactions de la carte}
Ici l'utilisateur pourra visualiser l'historique de la carte qu'il avait sélectionné.
Il pourra ensuite
\begin{enumerate}
    \item Retourner à la liste des cartes
    \item Exporter son historique -> Retourner directement à la liste des cartes
\end{enumerate}
\end{document}