\paragraph{Fenêtres disponibles}
\begin{itemize}
\item Auth ;
\item Sign in ;
\item Sign up ;
\item Change password ;
\item Change language ;
\item Main screen ;
\item Client ;
\item Details ;
\item Export data ;
\item Create client account ;
\item Requests ;
\item Transfer permission requests ;
\item Portfolio requests ;
\item Manage data ;
\item Import data.
\end{itemize}


\subparagraph{Fenêtre \emph{Auth}}
\access{en ouvrant l'application sans être connecté, en cliquant sur les boutons \emph{Back} des fenêtres \emph{Sign in} et \emph{Sign up} ou en cliquant sur le bouton \emph{Sign out} de la fenêtre \emph{Main screen}.}
\content{les boutons \emph{Sign in}, \emph{Sign up} et \emph{Language}.}
\begin{itemize}
\item \navbutton{Sign in}{Sign in} ;
\item \navbutton{Sign up}{Sign up} ;
\item \navbutton{Language}{Change language}.
\end{itemize}


\subparagraph{Fenêtre \emph{Sign in}}
\access{en cliquant sur le bouton \emph{Sign in} de la fenêtre \emph{Auth} ou en créant un compte sur la fenêtre \emph{Sign up}.}
\content{les boutons \emph{Back}, \emph{Sign in} et \emph{Language}, le champ de texte \emph{SWIFT...} et le champ de mot de passe \emph{Password...} .}
\begin{itemize}
\item \navbutton{Back}{Auth} ;
\item Bouton \emph{Sign in} : si les identifiants sont corrects, connecte l'utilisateur et l'envoie sur la fenêtre \emph{Main screen}, sinon, affiche une erreur ;
\item \navbutton{Language}{Change language} ;
\item \textfield{SWIFT...}{le code SWIFT de la banque} ;
\item \passwordfield{Password...}{le mot de passe du compte de la banque}.
\end{itemize}


\subparagraph{Fenêtre \emph{Sign up}}
\access{en cliquant sur le bouton \emph{Sign up} de la fenêtre \emph{Auth}, les champs de texte \emph{SWIFT...}, \emph{City...}, \emph{Country...}, \emph{Name...} et \emph{Password...}, les champs de mot de passe \emph{Password...} et \emph{Confirm password...}, le label \emph{Favorite language}, le menu déroulant \emph{Language} et la case à cocher \emph{Confirmation}.}
\begin{itemize}
\item \navbutton{Back}{Auth} ;
\item Bouton \emph{Sign up} : si le code SWIFT est au format correct et disponible, les champs de texte \emph{City...}, \emph{Country...} et \emph{Name...} sont remplis, le mot de passe est confirmé, la langue favorite est sélectionnée et la case est cochée, enregistre un nouveau profil dans la base de données et envoie l'utilisateur sur la fenêtre \emph{Sign in} avec le champ de texte \emph{SWIFT...} pré-complété ;
\item \navbutton{Language}{Change language} ;
\item \textfield{SWIFT...}{le code SWIFT de la banque} ;
\item \textfield{City...}{la ville dans laquelle la banque est basée} ;
\item \textfield{Country...}{le pays dans lequel la banque est basée} ;
\item \textfield{Name...}{le nom de la banque} ;
\item \passwordfield{Password...}{un mot de passe} ;
\item \passwordfield{Confirm password...}{la confirmation du mot de passe} ;
\item Le menu déroulant \emph{Language} : permet à l'utilisateur de sélectionner sa langue préférée ;
\item La case à cocher \emph{Confirmation} : demande à l'utilisateur de confirmer qu'il a sauvegardé son mot de passe de manière locale, sécurisée et permanente.
\end{itemize}


\subparagraph{Fenêtre \emph{Change password}}
\access{en cliquant sur le bouton \emph{Change password} de la fenêtre \emph{Main screen}.}
\content{les boutons \emph{Back}, \emph{Change password} et \emph{Language} et les champ de mot de passe \emph{Current password}, \emph{New password} et \emph{Confirm new password} et la case à cocher \emph{Confirmation}.}
\begin{itemize}
\item \navbutton{Back}{Main screen} ;
\item Le bouton \emph{Change password} : change le mot de passe lié à l'utilisateur si la case à cocher est cochée, que l'ancien mot de passe est correct et que la confirmation du nouveau mot de passe est correcte ;
\item \navbutton{Language}{Change language} ;
\item \passwordfield{Current password}{son mot de passe actuel} ;
\item \passwordfield{New password}{le nouveau mot de passe désiré} ;
\item \passwordfield{Confirm new passsword}{la confirmation du nouveau mot de passe} ;
\item La case à cocher \emph{Confirmation} : demande à l'utilisateur de confirmer qu'il a mis à jour son mot de passe précédemment enregistré de manière locale, sécurisée et permanente.
\end{itemize}


\subparagraph{Fenêtre \emph{Change language}}
\access{en cliquant sur le bouton \emph{Language} de la fenêtre \emph{Auth}, \emph{Sign in}, \emph{Sign up}, \emph{Reset password} ou \emph{Main screen}.}
\content{les boutons \emph{Back}, \emph{Add...} et \emph{Confirm}, le label \emph{Choose a language} et la liste \emph{Available languages}.}
\begin{itemize}
\item \navbutton{Back}{précédente} ;
\item Le bouton \emph{Add...} : permet de sélectionner un fichier JSON correspondant à une langue afin de l'ajouter à la liste de langues disponibles ;
\item \navbutton{Confirm}{précente} et modifie la langue de l'application par la nouvelle langue sélectionnée ;
\item La liste \emph{Available languages} : contient une liste de toutes les langues disponibles et permet à l'utilisateur d'en sélectionner une.
\end{itemize}


\subparagraph{Fenêtre \emph{Main screen}}
\access{en se connectant à son compte de la fenêtre \emph{Sign in}, en cliquant sur le bouton \emph{Back} de la fenêtre \emph{Clients}, en cliquant sur les boutons \emph{Change password} ou \emph{Back} de la fenêtre \emph{Change password}, en cliquant sur le bouton \emph{Back} de la fenêtre \emph{Financial products} ou en cliquant sur le bouton \emph{Back} de la fenêtre \emph{Manage data}.}
\content{les boutons \emph{Sign out}, \emph{Language}, \emph{Clients}, \emph{Manage requests}, \emph{Manage data} et \emph{Change password}.}
\begin{itemize}
\item \navbutton{Sign out}{Auth} et déconnecte l'utilisateur ;
\item \navbutton{Language}{Change language} ;
\item \navbutton{Clients}{Clients} ;
\item \navbutton{Manage requests}{Requests} ;
\item \navbutton{Manage data}{Manage data};
\item \navbutton{Change password}{Change password}.
\end{itemize}


\subparagraph{Fenêtre \emph{Clients}}
\access{en cliquant sur le bouton \emph{Clients} de la fenêtre \emph{Main screen}, en cliquant sur le bouton \emph{Back} de la fenêtre \emph{Details}, en cliquant sur le bouton \emph{Back}, \emph{Export to JSON format} ou \emph{Export to CSV format} de la fenêtre \emph{Export data} ou en cliquant sur le bouton \emph{Back} de la fenêtre \emph{Add client}.}
\content{les boutons \emph{Back}, \emph{Search}, \emph{Export data...}, \emph{Add client} et \emph{Details}, les labels \emph{Choose a client}, \emph{Search name or ID}, \emph{Sort by}, le champ de texte \emph{Enter search...}, le menu déroulant \emph{Sort} et la liste \emph{Clients}.}
\begin{itemize}
\item \navbutton{Back}{Main screen} ;
\item La liste \emph{Clients} : affiche la liste des clients associés à l'institution (c'est-à-dire ceux étant titulaires ou co-titulaires d'un compte dans cette institution) ;
\item \navbutton{Export data...}{Export data} ;
\item \navbutton{Add client}{Add client} ;
\item \navbutton{Details}{Details} ;
\item Le bouton \emph{Search} : effectue une recherche en fonction des critères entrés par l'utilisateur ;
\item \textfield{Enter search...}{sa recherche} ;
\item Le menu déroulant \emph{Sort} : permet à l'utilisateur de choisir un critère de tri des résultats.
\end{itemize}

\subparagraph{Fenêtre \emph{Add client}}
\access{en cliquant sur le bouton \emph{Add client} de la fenetre \emph{Clients}.}
\content{les boutons \emph{Back} et \emph{Add client}, le label \emph{Client NRN} et le champ de texte \emph{NRN...}.}
\begin{itemize}
\item \navbutton{Back}{Clients} ;
\item Le bouton \emph{Add client} : ajoute un client dans la table des clients de la banque dans la base de données ;
\item \textfield{NRN...}{le numéro de registre national du client}.
\end{itemize}


\subparagraph{Fenêtre \emph{Details}}
\access{en cliquant sur le bouton \emph{Details} de la fenêtre \emph{Clients}, en cliquant sur les boutons \emph{Back}, \emph{Export to JSON format} ou \emph{Export to CSV formate} de la fenêtre \emph{Export data} ou en cliquant sur le bouton \emph{Back} de la fenêtre \emph{Create account}.}
\content{les boutons \emph{Back}, \emph{Remove client}, \emph{Search}, \emph{Export data...}, \emph{Create account} et \emph{Close account}, les labels \emph{Choose an account}, \emph{Client}, \emph{Search IBAN} et \emph{Sort by}, le champ de texte \emph{Enter search...}, le menu déroulant \emph{Sort} et la liste \emph{Accounts}.}
\begin{itemize}
\item \navbutton{Back}{Clients} ;
\item Le bouton \emph{Remove client} : enlève le client de la banque ;
\item Le bouton \emph{Search} : effectue une recherche en fonctino des critères entrés par l'utilisateur ;
\item \navbutton{Export data...}{Export data} ;
\item \navbutton{Create account}{Create account} ;
\item Le bouton \emph{Close account} : supprime le compte de ce client ;
\item \textfield{Enter search...}{sa recherche} ;
\item Le menu déroulant \emph{Sort} : permet à l'utilisateur de choisir un critère de tri des résultats.
\end{itemize}


\subparagraph{Fenêtre \emph{Export data}}
\access{en cliquant sur le bouton \emph{Export data...} de la fenêtre \emph{Clients} ou \emph{Details} ou en cliquant sur le bouton \emph{Export all client data} de la fenêtre \emph{Manage data}.}
\content{les boutons \emph{Back}, \emph{Choose path...}, \emph{Export to JSON format} et \emph{Export to CSV format} et les labels \emph{Export location} et \emph{Selected path}.}
\begin{itemize}
\item \navbutton{Back}{précédente} ;
\item Le bouton \emph{Choose path...} : ouvre l'explorateur de fichier afin de déterminer le chemin de destination ;
\item Le bouton \emph{Export to JSON format} : exporte l'historique au format JSON ;
\item Le bouton \emph{Export to CSV format} : exporte l'historique au format CSV.
\end{itemize}


\subparagraph{Fenêtre \emph{Create client account}}
\access{en cliquant sur le bouton \emph{Create account} de la fenêtre \emph{Details}.}
\content{les boutons \emph{Back} et \emph{Create account}, les labels \emph{Account type}, \emph{Client name}, \emph{IBAN} et \emph{Client ID} et le menu déroulant \emph{Type}.}
\begin{itemize}
\item \navbutton{Back}{Details} ;
\item \navbutton{Create account}{Details} et crée le compte ;
\item Le menu déroulant \emph{Type} : permet à l'utilisateur d'entrer le type de compte à créer.
\end{itemize}


\subparagraph{Fenêtre \emph{Requests}}
\access{en cliquant sur le bouton \emph{Manage requests} de la fenêtre \emph{Main screen}, en cliquant sur le bouton \emph{Back} de la fenêtre{Transfer permission requests} ou en cliquant sur le bouton \emph{Back} de la fenêtre \emph{Portfolio requests}.}
\content{les boutons \emph{Back}, \emph{Transfer permission requests} et \emph{Portfolio requests}.}
\begin{itemize}
\item \navbutton{Back}{Main screen} ;
\item \navbutton{Transfer permission requests}{Transfer permission requests} ;
\item \navbutton{Portfolio requests}{Portfolio requests}.
\end{itemize}


\subparagraph{Fenêtre \emph{Transfer permission requests}}
\access{en cliquant sur le bouton \emph{Transfer permission requests} de la fenêtre \emph{Requests}.}
\content{les bouton \emph{Back}, \emph{Deny} et \emph{Approve}, le label \emph{Choose a request} et la liste \emph{Requests}.}
\begin{itemize}
\item \navbutton{Back}{Requests} ;
\item Le bouton \emph{Deny} : refuse la requête sélectionnée ;
\item Le bouton \emph{Approve} : accepté la requête sélectionnée ;
\item La liste \emph{Requests} : liste les requêtes des utilisateurs.
\end{itemize}


\subparagraph{Fenêtre \emph{Portfolio requests}}
\access{en cliquant sur le bouton \emph{Portfolio requests} de la fenêtre \emph{Requests}.}
\content{les bouton \emph{Back}, \emph{Deny} et \emph{Approve}, le label \emph{Choose a request} et la liste \emph{Requests}.}
\begin{itemize}
\item \navbutton{Back}{Requests} ;
\item Le bouton \emph{Deny} : refuse la requête sélectionnée ;
\item Le bouton \emph{Approve} : accepté la requête sélectionnée ;
\item La liste \emph{Requests} : liste les requêtes des utilisateurs.
\end{itemize}


\subparagraph{Fenêtre \emph{Manage data}}
\access{en cliquant sur le bouton \emph{Manage data} de la fenêtre \emph{Main screen} ou en cliquant sur le bouton \emph{Back} de la fenêtre \emph{Import data}.}
\content{les boutons \emph{Back}, \emph{Import data} et \emph{Export all client data}.}
\begin{itemize}
\item \navbutton{Back}{Main screen} ;
\item \navbutton{Import data}{Import data} ;
\item \navbutton{Export all client data}{Export data}.
\end{itemize}


\subparagraph{Fenêtre \emph{Import data}}
\access{en cliquant sur le bouton \emph{Import data} de la fenêtre \emph{Manage data}.}
\content{les boutons \emph{Back}, \emph{Choose file...} et \emph{Import file} et les labels \emph{Choose file} et \emph{Selected file}.}
\begin{itemize}
\item \navbutton{Back}{Manage data} ;
\item le bouton \emph{Choose file...} : ouvre l'explorateur de fichier afin de déterminer le fichier à importer ;
\item Le bouton \emph{Import file} : importe les données des fichiers.
\end{itemize}