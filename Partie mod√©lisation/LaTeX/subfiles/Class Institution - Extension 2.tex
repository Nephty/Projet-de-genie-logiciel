\paragraph{Introduction :} Comme dit dans la partie client, des changements ont été effectué
entre les deux applications. Ce diagramme est donc légèrement différent du diagramme de l'application client.
Cependant, certains changement dans ce diagramme sont également dans la partie client et y sont déjà expliqué.

\paragraph{Partie logique}
\begin{itemize}
    \item La classe \emph{ClientManager} a été modifié afin de permettre aux institutions d'ajouter un sous-compte grace à la méthode \emph{addSubAccount()}.
    \item La classe \emph{DataManager} a également été modifiée afin de permettre aux institutions d'ajouter ou de modifier les frais de virement vers un pays hors zone SEPA.
\end{itemize}


\paragraph{Partie API}
Comme dans la partie client, peu de changement ont été effectué dans cette partie.\\
Uniquement 2 classes ont été ajoutée : 
\begin{itemize}
    \item Une classe \emph{SubAccountController} a été ajoutée afin de permettre aux institution d'ajouter et de modifier un sous-compte.
    \item Une classe \emph{CountryFeeController} a également été ajoutée afin de pouvoir ajouter/modifier des frais sur un pays hors zone SEPA.
\end{itemize}
Les méthodes de ces classes seront appelée par les classe \emph{ClientManager} et \emph{DataManager} afin qu'uniquement la partie logique n'appelle ces classes et de garder le schéma assez clair.

\paragraph{Partie GUI}
Encore une fois, une classe par fenêtre a été ajoutée tout en respectant le design pattern singleton.