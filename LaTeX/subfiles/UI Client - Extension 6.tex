\paragraph{Nouvelles fenêtres disponibles}
\begin{itemize}
\item QR codes ;
\item Generate QR code ;
\item Read QR code ;
\item Suspicious transactions history ;
\item Due payments.
\end{itemize}

\paragraph{Fenêtres modifiées}
\begin{itemize}
\item Main screen. % new access, new content
\end{itemize}


\subparagraph{Fenêtre \emph{QR codes}}
\access{en cliquant sur le bouton \emph{QR codes} de la fenêtre \emph{Main screen}.}
\content{les boutons \emph{Back}, \emph{Generate} et \emph{Read}.}
\begin{itemize}
\item \navbutton{Back}{Main screen} ;
\item \navbutton{Generate}{Generate QR code} ;
\item \navbutton{Read}{Read QR code}.
\end{itemize}


\subparagraph{Fenêtre \emph{Generate QR code}}
\access{en cliquant sur le bouton \emph{Generate} de la fenêtre QR code.}
\content{les boutons \emph{Back}, \emph{Choose path...} et \emph{Generate}, les labels \emph{Amount}, \emph{Message}, \emph{Export location} et \emph{Selected path} et les champs de texte \emph{Amount} et \emph{Message}.}
\begin{itemize}
\item \navbutton{Back}{QR codes} ;
\item Le boutton \emph{Choose path...} : ouvre l'explorateur de fichier afin de déterminer le chemin de destination ;
\item Le bouton \emph{Generate} : génère un QR code si le montant est un nombre supérieur à zéro, l'IBAN est correctement formatté et la destination est choisie ;
\item \textfield{Amount}{le montant} ;
\item \textfield{Message}{la communication}.
\end{itemize}


\subparagraph{Fenêtre \emph{Read QR code}}
\access{en cliquant sur le bouton \emph{Read} de la fenêtre \emph{QR codes}.}
\content{les boutons \emph{Back}, \emph{Choose file...} et \emph{Pay}, les labels \emph{Choose file}, \emph{Selected file}, \emph{Account} et \emph{QR code state} et le menu déroulant \emph{Account}.}
\begin{itemize}
\item \navbutton{Back}{QR codes} ;
\item Le bouton \emph{Choose file...} : ouvre l'explorateur de fichier afin de déterminer le fichier à lire ;
\item Le bouton \emph{Pay} : effectue le paiement si le QR code est valide ;
\item Le menu déroulant \emph{Account} : permet à l'utilisateur de sélectionner le compte à partir duquel retirer l'argent.
\end{itemize}


\subparagraph{Fenêtre \emph{Suspicious transactions history}}
\access{en cliquant sur le bouton \emph{Suspicious transactions} de la fenêtre \emph{Main screen}.}
\content{les boutons \emph{Back}, \emph{Intentional} et \emph{Suspicious}, le label \emph{Choose a suspicion} et la liste \emph{Suspicious transaction history}.}
\begin{itemize}
\item \navbutton{Back}{Main screen} ;
\item Le bouton \emph{Intentional} : permet à l'utilisateur de marquer une transaction comme intentionnelle et d'infirmer la suspicion ;
\item Le bouton  \emph{Suspicious} : permet à l'utilisateur de marquer une transaction comme suspecte et de confirmer la suspicion ;
\item La liste Suspicious transaction history : contient tout l'historique des transactions marquées comme suspectes.
\end{itemize}


\subparagraph{Fenêtre \emph{Due payments}}
\access{en cliquant sur le bouton \emph{Due payments} de la fenêtre \emph{Main screen}.}
\content{les boutons \emph{Back}, \emph{Pay} et \emph{Pay contactless}, les labels \emph{Choose a payment} et \emph{Choose an account}, le menu déroulant \emph{Account} et la liste \emph{Due payments}.}
\begin{itemize}
\item \navbutton{Back}{Main screen} ;
\item Le bouton \emph{Pay} : permet à l'utilisateur de payer "avec contact" avec le compte sélectionné, c'est-à-dire entrer son compte PIN afin de valider la transaction ;
\item Le bouton \emph{Pay contactless} : permet à l'utilisateur de payer "sans contact" avec le compte sélectionné, c'est-à-dire valider la transaction sans entrer son code PIN, à condition que la limite ne soit pas dépassée ;
\item Le menu déroulant \emph{Account} : permet à l'utilisateur de sélectionner le compte à partir duquel effectuer le paiement ;
\item La liste \emph{Due payments} : affiche tous les paiements trouvés dans le fichier dédié et permet à l'utilisateur d'en sélecionner un à effectuer.
\end{itemize}


\subparagraph{Fenêtre \emph{Main screen}}
\ \\ \noindent\textbf{Nouvel accès :} en cliquant sur le bouton \emph{Back} des fenêtre \emph{QR codes}, \emph{Suspicious transactions history} ou \emph{Due payments}.
\ \\ \textbf{Nouveau contenu :} les boutons \emph{Due payments}, \emph{Suspicious transactions} et \emph{QR codes}.
\begin{itemize}
\item \navbutton{Due payments}{Due payments} ;
\item \navbutton{Suspicious transactions}{Suspicious transactions history} ;
\item \navbutton{QR codes}{QR codes}.
\end{itemize}

