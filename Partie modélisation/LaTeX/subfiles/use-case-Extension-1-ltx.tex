\documentclass{article}
\newcounter{rownumbers}
\newcommand\rownumber{\stepcounter{rownumbers}\arabic{rownumbers}}
\usepackage[utf8]{inputenc}
\usepackage{array}
\usepackage{geometry}
\usepackage{hyperref}

\begin{document}

Les fonctionnalités requises de mon extension m'ont amené à rajouter
des cas d'utilisation supplémentaires à ce diagramme. Il y a la possibilité
de créer une carte, de bloquer une carte en cas de perte ou si l'on ne souhaite plus utiliser
la carte. On peut aussi modifier les paramètres de la cartes cela inclus changer le plafond de
la carte de crédit ou désactiver les transactions vers l'étranger par exemple. La
visualisation de l'historique et l'export fonctionne de la même manière que dans les fonctions
de base car les données manipulées sont des transactions dans tous les cas.
Il y a finalement l'option de payer avec une carte qui est disponible.

\end{document}