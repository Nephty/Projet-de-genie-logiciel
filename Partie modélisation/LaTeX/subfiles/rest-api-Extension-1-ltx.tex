\documentclass{article}
\usepackage[utf8]{inputenc}
\usepackage[T1]{fontenc}
\usepackage[french]{babel}
\usepackage{amsmath,amsfonts,amssymb,amsthm}
\usepackage[margin=2.5cm]{geometry}

\begin{document}


\paragraph{Introduction}

Ce diagramme décrit les ajouts et modifications apportés à l'api pour l'extension de
gestion de cartes.

\paragraph{\textbackslash transaction}
   
    \begin{enumerate}
        \item \textbf{GET}: Un paramètre de recherche \emph{card-id} a été rajouté. Si il est spécifié, l'api saura que la
                            transaction est effectuée par une carte. Elle ignorera alors le paramètre \emph{wallet-iban} et 
                            retournera la liste des transactions liées à la carte.
        \item \textbf{POST}: Un paramètre de recherche \emph{card-id} a été rajouté. Si il est spécifié, l'api saura que la
                            transaction est effectuée par une carte
    \end{enumerate}

\paragraph{\textbackslash card}

    \begin{enumerate}
        \item \textbf{DELETE}: Le paramètre card-id est requis cette méthode supprime simplement une carte.
    \end{enumerate}

    \subparagraph{\textbackslash card\textbackslash debit-card}
        \begin{enumerate}
            \item \textbf{POST}: Création d'une nouvelle carte de débit .
            \item \textbf{PUT}: Modifier les paramètres de la carte de débit. \emph{card-id} est la clé primaire. Les autres paramètres
                                sont ceux qui seront modifiés.
            \item \textbf{GET}: Retourne une liste d'instances de DebitCard liées à un user \emph{user-id}
        \end{enumerate}
    
    \subparagraph{\textbackslash card\textbackslash credit-card}
        \begin{enumerate}
            \item \textbf{POST}: Création d'une nouvelle carte de crédit.
            \item \textbf{PUT}: Modifier les paramètres de la carte de crédit. \emph{card-id} est la clé primaire. Les autres paramètres
                                sont ceux qui seront modifiés.
            \item \textbf{GET}: Retourne une liste d'instances de CreditCard liées à un user \emph{user-id}
        \end{enumerate}


\end{document}