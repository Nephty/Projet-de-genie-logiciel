\documentclass{article}
\usepackage[utf8]{inputenc}
\usepackage[T1]{fontenc}
\usepackage[french]{babel}
\usepackage{amsmath,amsfonts,amssymb,amsthm}
\usepackage[margin=2.5cm]{geometry}


\begin{document}

\paragraph{Introduction}
Ce document parle des modifications apportées au diagramme de classe
de la partie commune pour l'extensions de gestions de cartes.


\paragraph{Partie Logique}

La classe abstraite \emph{AbstractCard} décrit le comportement commun pour entre les cartes
de crédit et débit. Comme la gestion de l'historique et le blocage d'une carte. Les classes
\emph{CreditCard} et \emph{DebitCard} spécifie les différences entre ces 2 types de cartes
notamment avec les transaction et la récupération des données de la cartes. De plus la carte
de Crédit a 2 attributs supplémentaires \emph{maxAmount} et \emph{availableAmount} qui
indique respectivement le plafond de la carte ainsi que le montant restant pour le mois.
L'énumération CardBrand contient la liste des différentes marques de cartes utilisée
\noindent
\textbf{Classes rajoutées}:
\begin{itemize}
    \item AbstractCard
    \item CreditCard
    \item DebitCard
    \item CardBrand
\end{itemize}


\paragraph{Patie GUI}
Pour chaque nouvelle scène, une classe a été rajoutée. Ces classes respectent toujours le
design pattern Singleton
\newline
\textbf{Classes rajoutées}:
\begin{itemize}
    \item CardPayScene
    \item CardManagementScene
    \item AddCreditCardScene
    \item AddDebitCardScene
    \item CreditCardScene
    \item DebitCardScene
\end{itemize}

\paragraph{Partie API}

Pour chaque nouvel endpoint, une classe a été rajoutée les méthodes correspondent aux méthodes
HTTP disponibles à cet endpoint.
\newline
\textbf{Classes rajoutées}:
\begin{itemize}
    \item CardController
    \item DebitCardController
    \item CreditCardController
\end{itemize}

\end{document}