\paragraph{Structure} Le diagramme de classe de l'application client suit globalement la même structure que celui de la partie commune. Des ajouts ont été faits au GUI : de nouvelles fenêtres requises par les nouvelles fonctionnalités sont disponibles et respectent toujours le design pattern Singleton. Un utilitaire (classe \emph{QRCode}) a été ajouté à la partie logique, ainsi qu'un ensemble d'attributs qui serviront notamment à la détection de fraude et à l'imposition de limites par l'institution financière. Trois énumérations ont aussi été ajoutées : \emph{SEPACountries}, \emph{Country} et \emph{State}, servant respectivement à identifier les pays faisant partie de la zone SEPA, les pays à partir desquels on peut effectuer une transaction et l'état d'une transaction. Quelques méthodes ont été ajoutés à l'API afin d'obtenir certaines données plus facilement, sans devoir implémenter de nouvelles méthodes dans le code de l'application.