\documentclass{article}
\usepackage[utf8]{inputenc}
\usepackage[T1]{fontenc}
\usepackage[french]{babel}
\usepackage{amsmath,amsfonts,amssymb,amsthm}
\usepackage[margin=2.5cm]{geometry}

\newcommand{\navbutton}[2]{Le bouton \emph{#1} : envoie l'utilisateur sur la scène \emph{#2}}
\newcommand{\textfield}[2]{Le champ de texte \emph{#1} : permet à l'utilisateur d'entrer #2}
\newcommand{\passwordfield}[2]{Le champ de mot de passe \emph{#1} : permet à l'utilisateur d'entrer #2 de manière discrète}
\newcommand{\access}[1]{ \noindent\textbf{Accès} : #1 \\}
\newcommand{\modif}[1]{\textbf{Contenu modifié/ajouté} : #1}
\newcommand{\content}[1]{\textbf{Contenu} : #1}

\begin{document}

\paragraph{Introduction}
\noindent
Ce document décrit l'utilisation de l'interface graphique de l'application institution en se basant sur la maquette de l'interface. \\

\paragraph{Scènes modifiées}
\begin{itemize}
\item Requests ;
\end{itemize}

\paragraph{Scène \emph{Requests}}
Elle assume le rôle d'approuver ou refuser une demande. Peut importe son type.
Cela comprends notamment les demandes pour bloquer une carte ou en créer une nouvelle.
\newline
\content{}
\begin{itemize}
    \item Le bouton \emph{Deny} : Va communiquer à l'api que la requête a été refusée;
    \item Le bouton \emph{Approve} : Va communiquer à l'api que la requête a été acceptée;
    \item \navbutton{Back}{Main Screen}
    \item La list \emph{RequestList} : Contient la liste des requêtes de la banque;
\end{itemize}

\end{document}