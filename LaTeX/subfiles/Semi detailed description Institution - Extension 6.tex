\documentclass{article}
\usepackage[utf8]{inputenc}
\usepackage[T1]{fontenc}
\usepackage[french]{babel}
\usepackage{amsmath,amsfonts,amssymb,amsthm}

% 1 = Créer une demande de paiement via QR code
% 2 = Effectuer un paiement via QR code
% 3 = Voir la liste des paiements à effectuer
% 4 = Sélectionner un paiement à effectuer
% 5 = Payer sans contact
% 6 = Payer avec contact
% 7 = Voir l'historique des transactions suspectes
% 8 = Confirmer une suspicion de fraude
% 9 = Infirmer une suspicion de fraude
% 10 = Notifier le client d'une transaction suspecte
% 11 = Bloquer le compte d'un client



\begin{document}



\begin{table}

\begin{tabular}{|c|p{11cm}|}
\hline
Acteur & Serveur \\
\hline
Description & Permet de bloquer un compte de l'utilisateur \\
\hline
Préconditions & Une action (peu importe sa nature) ayant pour conséquence de bloquer le compte d'un utilisateur a eu lieu \\
\hline
Postconditions & Le compte est bloqué \\
\hline
Scénario principal & \begin{enumerate}
\item Une notification est envoyée au client ;
\item Le compte en question est désormais inutilisable.
\end{enumerate} \\
\hline
Scénario alternatif & Aucun \\
\hline
Trigger & Lorsqu'une action demandant à bloquer un compte a lieu \\
\hline
Fréquence d'utilisation & Très rare \\
\hline
\end{tabular}

\caption{Modifier les limites}

\end{table}


\end{document}
