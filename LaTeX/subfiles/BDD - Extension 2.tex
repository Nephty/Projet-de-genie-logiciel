Pour cette extension, de nombreux changements ont été effectués au niveau du diagramme d'entité relation.\\
Tout d'abord, chaque compte contient plusieurs subaccounts qui sont différencier par le type de devise. Ce qui permet de créer des comptes multi-devises. 
Si un compte n'est pas multi-devise, il ne possèdera qu'un seul subaccount.\\
L'ajout de ces SubAccount implique quelques changement de l'entité \emph{TransactionLog} car il faut effectuer le virement depuis un subaccount vers un autre subaccount.
C'est pour cela que les deux attributs \textit{CurrencyI\_Used} et \textit{CurrencyID\_Recipient} ont été ajouté car ils permettent de décrire de quel devise a été envoyé l'argent et vers quelle devise.
Un attribut est également ajouté afin de connaitre le taux de conversion entre les deux devise au moment de la transaction.\\
Une entité \emph{CountryFee} a également été ajoutée afin de connaitre les frais de virement internationaux pour chaque banque. 
Si un utilisateur souhaite effectuer un virement vers un pays qui n'est pas dans la liste de sa banque, le virement ne peut pas être effectué.