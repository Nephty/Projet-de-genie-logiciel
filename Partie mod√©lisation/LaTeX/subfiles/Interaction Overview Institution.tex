\paragraph{Accès à application institution financière, en tant qu'institution financière} La seconde partie n'est accessible qu'en se connectant avec les identifiants d'un compte enregistré dans la base de données. Lors de la connexion, l'utilisateur (le responsable de l'institution, l'économe, le gestionnaire...) arrive sur l'écran principal, et peut, depuis celui-ci, accéder à toutes les fonctionnalités. Après s'être connecté, l'utilisateur peut effectuer plusieurs actions :
\begin{enumerate}
\item Afficher la liste des clients de l'institution ;
\item Gérer les données de l'institution ;
\item Accéder à la liste des demandes faites auprès de l'institution.
\end{enumerate}
\begin{scriptsize}
\textit{\emph{Quitter l'application}, \emph{Changer la langue} et \emph{Se déconnecter} sont considéré comme des comportements triviaux.}
\end{scriptsize}


\subparagraph{Afficher la liste des clients de l'institution}
L'utilisateur peut accéder à la liste des clients et peut ajouter un client, exporter les données de ses clients ou en sélectionner un afin d'accéder aux détails.


\subparagraph{Sélectionner un client}
Après avoir sélectionné un client, l'utilisateur peut supprimer le client de la base de données, clôturer un compte ou en ouvrir un nouveau (de n'importe quel type).
\begin{scriptsize}
\textit{Ces trois comportements sont considérés comme triviaux.}
\end{scriptsize}


\subparagraph{Ajouter un client}
L'utilisateur peut ajouter un nouveau client en spécifiant les informations nécéssaires.


\subparagraph{Exporter les données de ses clients (implémente le tri et la recherche)}
Lorsque l'utilisateur effectue une recherche ou un tri de ses clients, une liste différente de clients sera affichée à l'écran. Il pourra alors exporter cette liste "personnalisée" au format JSON ou CSV, qui contiendra les informations à disposition de la banque à propos de ses clients.


\subparagraph{Gérer les données de l'institution}
Grâce à cette fonctionnalité, l'utilisateur connecté peut importer des données ou exporter toutes les données de tous les clients.


\subparagraph{Importer des données}
L'import de données peut servir, par exemple, à restaurer une liste de client dans la base de données pour la mettre à disposition de la banque via l'interface graphique. L'utilisateur peut choisir un fichier au format JSON qui sera lisible par l'application.


\subparagraph{Exporter toutes les données de tous les clients}
L'export de toutes les données des clients permet d'exporter toutes les données en un clic ; c'est un moyen alternatif, plus rapide, mais qui ne permet pas le tri ou la recherche de mots clés.


\subparagraph{Accéder à la liste des demandes}
L'utilisateur peut, finalement, accéder aux demandes, qui sont de deux types : demandes de permission de transferts et demande d'ouverture de compte. En sélectionnant l'une ou l'autre catégorie, il pourra approuver ou refuser une demande.