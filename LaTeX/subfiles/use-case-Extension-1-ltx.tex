\documentclass{article}
\newcounter{rownumbers}
\newcommand\rownumber{\stepcounter{rownumbers}\arabic{rownumbers}}
\usepackage[utf8]{inputenc}
\usepackage{array}
\usepackage{geometry}
\usepackage{hyperref}

\begin{document}
%newpage entre chaque tableau

%Nom de votre UseCase
\paragraph{Application Client}

\subparagraph{Voir la liste des cartes}
    \begin{table}[h]
        \begin{tabular}{|c|p{10cm}|}
        \hline
        Acteur principal& Utilisateur et Serveur \\
        \hline
        Description& Permet à un utilisateur d'afficher la liste des cartes \\
        \hline
        Préconditions& L'utilisateur doit être connecté  \\
        \hline
        PostConditions&  /    \\
        \hline
        Scénario principal& 
                \begin{enumerate}
                    \item L'utilisateur entre les données de la carte
                    \item L'application envoie une demande de création de carte à l'API
                    \item L'API répond que la demande de création de carte a été ajoutée ou qu'une erreur s'est produite
                \end{enumerate}     \\
        \hline
        Scénario alternatif& 
            \begin{enumerate}
                \item L'utilisateur entre les données de la carte
                \item L'application envoie une demande de création de carte à l'API
                \item L'API répond qu'une erreur s'est produite
                \item L'application affiche un message d'erreur
            \end{enumerate}     \\
        \hline
        Trigger&  Lorsque l'utilisateur appuie sur le bouton pour ajouter une carte   \\
        \hline
        Fréquence d'utilisation& Rarement utilisée  \\
        \hline
        \end{tabular}
    \end{table}

    \newpage

\subparagraph{Créer une carte}
    \begin{table}[h]
        \begin{tabular}{|c|p{10cm}|}
        \hline
        Acteur principal& Utilisateur et Serveur \\
        \hline
        Description& Permet à un utilisateur de créer une nouvelle carte \\
        \hline
        Préconditions& L'utilisateur doit être connecté et avoir introduit les données nécessaires
            à la création de la carte \\
        \hline
        PostConditions&  Une nouvelle carte a été crée\\
        \hline
        Scénario principal& 
                \begin{enumerate}
                    \item L'utilisateur entre les données de la carte
                    \item L'application envoie une demande de création de carte à l'API
                    \item L'API répond que la demande de création de carte a été ajoutée ou qu'une erreur s'est produite
                \end{enumerate}     \\
        \hline
        Scénario alternatif& 
            \begin{enumerate}
                \item L'utilisateur entre les données de la carte
                \item L'application envoie une demande de création de carte à l'API
                \item L'API répond qu'une erreur s'est produite
                \item L'application affiche un message d'erreur
            \end{enumerate}     \\
        \hline
        Trigger&  Lorsque l'utilisateur appuie sur le bouton pour ajouter une carte   \\
        \hline
        Fréquence d'utilisation& Rarement utilisée  \\
        \hline
        \end{tabular}
    \end{table}

    \newpage

\subparagraph{Bloquer une carte}
    \begin{table}[h]
        \begin{tabular}{|c|p{10cm}|}
        \hline
        Acteur principal& Utilisateur et Serveur    \\
        \hline
        Description&  Permet à un utilisateur de bloquer une carte  \\
        \hline
        Préconditions&    L'utilisateur doit avoir sélectionner une carte  \\
        \hline
        PostConditions&  L'API répond que la demande de blocage de la carte a été ajoutée ou qu'une erreur s'est produite    \\
        \hline
        Scénario principal& 
                \begin{enumerate}
                    \item Il actionne le bouton pour envoyer la demande à l'API
                    \item L'API répond que la demande de blocage de la carte a été ajoutée
                \end{enumerate}     \\
        \hline
        Scénario alternatif&  
        \begin{enumerate}
            \item Il actionne le bouton pour envoyer la demande à l'API
            \item L'API répond qu'une erreur s'est produite
            \item Un message d'erreur est affiché
        \end{enumerate}    \\
        \hline
        Trigger&   Lorsque l'utilisateur actionne le bouton Stop Card   \\
        \hline
        Fréquence d'utilisation&    Très rare  \\
        \hline
        \end{tabular}
    \end{table}

\newpage

\subparagraph{Modifier une carte}
    \begin{table}[h]
        \begin{tabular}{|c|p{10cm}|}
        \hline
        Acteur principal& Utilisateur et Serveur    \\
        \hline
        Description&  Permet à un utilisateur de modifier une carte  \\
        \hline
        Préconditions&    L'utilisateur doit avoir sélectionner une carte  \\
        \hline
        PostConditions&  L'API répond que la demande de modification de la carte a été ajoutée ou qu'une erreur s'est produite    \\
        \hline
        Scénario principal& 
                \begin{enumerate}
                    \item L'utilisateur modifie les données qu'il veut changer
                    \item Il actionne le bouton pour envoyer la demande à l'API
                    \item L'API répond que la demande de modification de la carte a été ajoutée
                \end{enumerate}     \\
        \hline
        Scénario alternatif&  
        \begin{enumerate}
            \item L'utilisateur modifie les données qu'il veut changer
            \item Il actionne le bouton pour envoyer la demande à l'API
            \item L'API répond qu'une erreur s'est produite
            \item Un message d'erreur est affiché
        \end{enumerate}    \\
        \hline
        Trigger&   Lorsque l'utilisateur actionne le bouton Submit après avoir modifier des données   \\
        \hline
        Fréquence d'utilisation&    Peu fréquent  \\
        \hline
        \end{tabular}
    \end{table}

\newpage

\subparagraph{Voir l'historique des transactions liées à la carte}
    \begin{table}[h]
        \begin{tabular}{|c|p{10cm}|}
        \hline
        Acteur principal& Utilisateur et Serveur    \\
        \hline
        Description&  Permet à un utilisateur d'exporter les transactions au format json  \\
        \hline
        Préconditions& L'utilisateur doit être connecté et avoir sélectionné une carte \\
        \hline
        PostConditions&  La liste des transactions est affichée   \\
        \hline
        Scénario principal& 
                \begin{enumerate}
                    \item L'utilisateur appuie sur le bouton history
                    \item Une demande pour recevoir l'historique des transactions est effectuée
                    \item L'API renvoie les données
                    \item Les données sont affichées
                \end{enumerate}     \\
        \hline
        Scénario alternatif&  
            \begin{enumerate}
                \item L'utilisateur appuie sur le bouton history
                \item Une demande pour recevoir l'historique des transactions est effectuée
                \item L'API répond qu'une erreur s'est produite
                \item L'application display une erreur et propose de refresh
            \end{enumerate}     \\
        \hline
        Trigger&   Lorsque l'utilisateur appuie sur le bouton pour voir l'historique  \\
        \hline
        Fréquence d'utilisation&  Moyennement Fréquent  \\
        \hline
        \end{tabular}
    \end{table}

\newpage

\subparagraph{Exporter l'historique des transactions liées à la carte}
    \begin{table}[h]
        \begin{tabular}{|c|p{10cm}|}
        \hline
        Acteur principal& Utilisateur et Serveur    \\
        \hline
        Description&  Permet à un utilisateur d'exporter les transactions au format json  \\
        \hline
        Préconditions& L'utilisateur doit être connecté et avoir la liste des transactions en local \\
        \hline
        PostConditions&  Un fichier json contenant les transactions de la carte est crée   \\
        \hline
        Scénario principal& 
                \begin{enumerate}
                    \item L'utilisateur appuie sur le bouton pour exporter
                    \item Un fichier json contenant les transactions de la carte est crée
                \end{enumerate}     \\
        \hline
        Scénario alternatif&  /    \\
        \hline
        Trigger&   Lorsque l'utilisateur appuie sur le bouton pour exporter  \\
        \hline
        Fréquence d'utilisation&    Peu fréquent  \\
        \hline
        \end{tabular}
    \end{table}

\newpage

\subparagraph{Payer avec une carte}
    \begin{table}[h]
        \begin{tabular}{|c|p{10cm}|}
        \hline
        Acteur principal& Utilisateur et Serveur    \\
        \hline
        Description&  Permet à un utilisateur de payer avec une carte  \\
        \hline
        Préconditions&   L'utilisateur doit avoir sélectionner une carte et avoir entré
             le compte destinataire ainsi que le montant(les autres infos non-nécessaires
             au paiement par carte seront auto-générées). \\
        \hline
        PostConditions&  L'API répond que la demande de transaction a été effectuée ou qu'une erreur s'est produite    \\
        \hline
        Scénario principal& 
                \begin{enumerate}
                    \item L'utilisateur sélectionne la carte avec laquelle il veut payer
                    \item Il entre le destinataire et le montant
                    \item Il appuie sur le bouton pour payer et envoit la demande à l'API
                    \item L'API répond que le demande a été faite avec succès
                    \item L'application affiche que la transaction a été effectuée
                \end{enumerate}     \\
        \hline
        Scénario alternatif&  
        \begin{enumerate}
            \item L'utilisateur sélectionne la carte avec laquelle il veut payer
            \item Il entre le destinataire et le montant
            \item Il appuie sur le bouton pour payer et envoit la demande à l'API
            \item L'API répond qu'une erreur s'est produite
            \item Un message d'erreur est affiché
        \end{enumerate}    \\
        \hline
        Trigger&   Lorsque l'utilisateur actionne le bouton Pay dans la scène Card Management   \\
        \hline
        Fréquence d'utilisation&   Fréquent  \\
        \hline
        \end{tabular}
    \end{table}

\newpage

\section{Institution financière}
Le use case est similaire à celui de la partie commune.

\end{document}