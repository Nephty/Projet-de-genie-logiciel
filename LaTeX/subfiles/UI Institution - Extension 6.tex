\paragraph{Nouvelles fenêtres disponibles}
\begin{itemize}
\item Limits.
\end{itemize}

\paragraph{Fenêtres modifiées}
\begin{itemize}
\item Details. % new access, new content
\end{itemize}


\subparagraph{Fenêtre \emph{Limits}}
\access{en cliquant sur le bouton \emph{Edit limits} de la fenêtre \emph{Details}.}
\content{les boutons \emph{Back} et \emph{Set limit}, les labels \emph{Client name}, \emph{Account}, \emph{Time scope}, \emph{Current limit} et \emph{New limit}, le champ de texte \emph{New limit...} et le menu déroulant \emph{Time scope}.}
\begin{itemize}
\item \navbutton{Back}{Details} ;
\item Le bouton \emph{Set limit} : modifie la limite maximale d'argent qu'il est possible de transférer à l'échelle de temps choisie dans le menu déroulant ;
\item Le menu déroulant \emph{Time scope} : permet de choisir l'échelle de temps pour laquelle il faut modifier la limite (at once, daily, weekly ou monthly).
\end{itemize}


\subparagraph{Fenêtre \emph{Details}}
\ \\ \noindent\textbf{Nouvel accès :}  en cliquant sur le bouton \emph{Back} de la fenêtre \emph{Limits}.
\ \\ \textbf{Nouveau contenu :}  le bouton \emph{Edit limits}.
\begin{itemize}
\item \navbutton{Edit limits}{Limits}.
\end{itemize}

