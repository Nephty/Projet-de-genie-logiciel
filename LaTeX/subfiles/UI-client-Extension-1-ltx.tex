\documentclass{article}
\usepackage[utf8]{inputenc}
\usepackage[T1]{fontenc}
\usepackage[french]{babel}
\usepackage{amsmath,amsfonts,amssymb,amsthm}
\usepackage[margin=2.5cm]{geometry}

\newcommand{\navbutton}[2]{Le bouton \emph{#1} : envoie l'utilisateur sur la scène \emph{#2}}
\newcommand{\textfield}[2]{Le champ de texte \emph{#1} : permet à l'utilisateur d'entrer #2}
\newcommand{\passwordfield}[2]{Le champ de mot de passe \emph{#1} : permet à l'utilisateur d'entrer #2 de manière discrète}
\newcommand{\access}[1]{ \noindent\textbf{Accès} : #1 \\}
\newcommand{\modif}[1]{\textbf{Contenu modifié/ajouté} : #1}
\newcommand{\content}[1]{\textbf{Contenu} : #1}

\begin{document}

\paragraph{Introduction}
\noindent
Ce document décrit les modifications effectuées à l'UI pour l'extension des cartes.
Les scènes modifiées sont en couleur orange. \\

\paragraph{Scènes modifiées}
\begin{itemize}
\item Sign in ;
\item Main screen ;
\item Transaction history ;
\item Cards Management ;
\item Credit Card ;
\item Debit Card ;
\item Card Pay
\item Add a Credit Card ;
\item Add a debit Card ;
\end{itemize}

\subparagraph{Scène \emph{Sign In}}
Une méthode de connexion alternative est ajoutée afin de permettre de se connecter
via sa carte de banque. Les textfields CardNumber et pin permettront d'entrer les
données de la carte avec laquelle on veut se connecter.



\subparagraph{Scène \emph{Main screen}}

\textbf{Contenu modifié/ajouté} : le bouton\emph{Cards}
\begin{itemize}
\item \navbutton{Cards}{Cards Management} ;
\end{itemize}


\subparagraph{Scène \emph{Transaction History}}
Cette scène se comporte de la meme manière que si l'on regardait les transactions
d'un certain compte. Je me suis permis de la recycler pour les cartes car que ce
soit pour un compte ou une carte les données utilisées pour représenter l'historique
sont des transactions sous forme de tableau. Les transactions d'une carte étant un
sous-ensemble des transactions d'un compte. La manière pour visualiser ou exporter
sera aussi identique.  


\subparagraph{Scène \emph{Cards Management}}
\content{les boutons\emph{Manage debit card}, \emph{Manage credit card} et \emph{Pay}.}
\begin{itemize}
\item \navbutton{Manage credit card}{Credit Card} ;
\item \navbutton{Manage debit card}{Debit Card} ;
\item \navbutton{Pay}{CardPay} ;
\end{itemize}

\subparagraph{Scène \emph{Credit Card}}
\begin{itemize}
\item \navbutton{add}{Add Credit Card} ;
\item La liste \emph{Cards list} : Affiche la liste des cartes de crédit ;
\item La combo box \emph{Change linked account} : permet de changer le compte lié à la carte. Elle 
contient la liste des comptes liés au profil;
\item La check box \emph{Foreign Transaction} : permet d'autoriser ou pas les transactions étrangères ;
\item La check box \emph{Negative transaction} : permet d'autoriser ou pas les virements mettant le compte en négatif ;
\item Le bouton \emph{Submit} : permet d'envoyer la requete de modification au serveur ;
\item Le bouton \emph{Stop Card} : Envoie une requête pour bloquer la carte ;
\item \navbutton{History}{Transaction History}
\item \navbutton{Back}{Cards Management}
\end{itemize}

\subparagraph{Scène \emph{Debit Card}}
\begin{itemize}
\item \navbutton{add}{Add Debit Card} ;
\item La liste \emph{Cards list} : Affiche la liste des cartes de débit ;
\item La combo box \emph{Change linked account} : permet de changer le compte lié à la carte. Elle 
contient la liste des comptes liés au profil;
\item La check box \emph{Foreign Transaction} : permet d'autoriser ou pas les transactions étrangères ;
\item La check box \emph{Negative transaction} : permet d'autoriser ou pas les virements mettant le compte en négatif ;
\item Le bouton \emph{Submit} : permet d'envoyer la requete de modification au serveur ;
\item Le bouton \emph{Stop Card} : Envoie une requête pour bloquer la carte ;
\item \navbutton{History}{Transaction History}
\item \navbutton{Back}{Cards Management}
\end{itemize}

\subparagraph{Scène \emph{Card Pay}}
\begin{itemize}
\item La liste \emph{Cards list} : Affiche la liste des cartes ;
\item Le text field \emph{Iban} : permet de spécifier le compte destinataire ;
\item Le text field \emph{Amount} : permet de spécifier le montant de la transaction ;
\item Le bouton \emph{Pay} : permet d'envoyer la requete de transaction au serveur ;
\item \navbutton{History}{Transaction History}
\item \navbutton{Back}{Cards Management}
\end{itemize}

\subparagraph{Scène \emph{Add a Debit Card}}
\begin{itemize}
\item La combo box \emph{Change linked account} : permet de changer le compte lié à la carte. Elle 
contient la liste des comptes liés au profil;
\item La combo box \emph{Brand} : permet de changer la marque de la carte;
\item La check box \emph{Foreign Transaction} : permet d'autoriser ou pas les transactions étrangères ;
\item La check box \emph{Negative transaction} : permet d'autoriser ou pas les virements mettant le compte en négatif ;
\item Le textfield \emph{Pin} : détermine le pin de la carte
\item Le bouton \emph{Submit} : permet d'envoyer la requete de modification au serveur et retourne vers Debit Card;
\item \navbutton{Cancel}{Debit Card}
\end{itemize}

\subparagraph{Scène \emph{Add a Credit Card}}
\begin{itemize}
\item La combo box \emph{Change linked account} : permet de changer le compte lié à la carte. Elle 
contient la liste des comptes liés au profil;
\item La combo box \emph{Brand} : permet de changer la marque de la carte;
\item La check box \emph{Foreign Transaction} : permet d'autoriser ou pas les transactions étrangères ;
\item La check box \emph{Negative transaction} : permet d'autoriser ou pas les virements mettant le compte en négatif ;
\item Le textfield \emph{Pin} : détermine le pin de la carte
\item Le textfield \emph{MaxAmount} : détermine le montant qui sera mis à disposition chaque mois sur la carte ;
\item Le bouton \emph{Submit} : permet d'envoyer la requete de modification au serveur et retourne vers Debit Card;
\item \navbutton{Cancel}{Debit Card}
\end{itemize}

\end{document}