Tout d'abord, veuillez remarquer que ce modèle représente l'entièreté du diagramme d'entité relation (application de base + extension), nous avez pris la décision de mettre le diagramme entier car 
il n'y aura qu'une seule base de donnée sur le serveur donc lors de la conception de nos extension, nous avons dû nous concerter afin que les modifications ne posent pas de problèmes. 
Les modifications de chaque extensions seront mises en évidences dans les parties concernées\\

Avec ce modèle de donnée, nous avons essayé de couvrir au maximum les besoins de l'application tout en gardant un schéma assez compréhensible.
Les deux entités \textit{USER} et \textit{BANKS} sont les entités centrale du schéma, elle permettent de gérer les données des utilisateurs ainsi que des banques.
Chaque user est identifié grace à son numéro de registre nationnal et chaque banque est identifiée grâce à son numéro SWIFT.\\
L"entité \textit{Accounts} permet de stocker tous les comptes de l'application. Il existe plusieurs type de compte où certains ont un rendement annuel, l'entité \textit{AccountType} permet de les différenciers.
Veuillez noter que les spécificités de chaque compte sera commun à toutes les banque par soucis de facilité.
Un compte est divisé en plusieurs sous-comptes permettant de faciliter l'implémentations de certains extension mais dans cette vesrion, un compte ne possèdera qu'un seul Subaccount.
L'entité \textit{AccountAccess} permet, quant à elle, de donner les accès à un utilisateur à un compte. Elle permet donc à un utilisateur d'être co-titulaire c'est à dire d'avoir accès à un compte sans être le titulaire de celui-ci.\\
Plusieurs entités sont utilisées pour les notification. Chaque notification est identifiée par un \textit{NotificationID} 
qui permet de connaitre tous les informations concernant cette notification (par exemple le type de notification ou l'état de celle-ci).
Deux entités sont créée, permettant de différencier les notifications des clients et les notifications des banques.\\
L'entité \textit{TransactionLog} permet de stocker l'historique des transaction effectuées ainsi que celles à effectuer. L'etats des transaction est connu grâce à l'attribut \textit{status}.
Remarquez également que nous ne connaissons pas le nouveau montant du compte après la transaction. Nous avons décidé de calculer les montants en ajoutant ou diminuant le montant des transactions au montant actuel du compte.
Cela prendra effectivement plus de temps à calculer cependant cela permet de rendre la table TransactionLog plus concise et permet aussi d'éviter certaines erreurs.
