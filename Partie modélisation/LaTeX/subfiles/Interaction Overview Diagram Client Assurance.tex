\documentclass[]{article}

\usepackage[utf8]{inputenc}
\usepackage[T1]{fontenc}
\usepackage[frenchb]{babel}
\usepackage{amsmath,amsfonts,amssymb,amsthm}
\usepackage{graphicx}

\begin{document}

    \begin{figure}[ht]
        \centering
        \includegraphics[scale=0.22]{img/InteractionDiagramClient.png}
        \caption{Interaction overview diagram de l'application client pour l'extension assurance}
        \label{fig1}
        \end{figure}

    \paragraph{}L’interaction overview diagram de l’extension assurance ajoute ses uses cases à celui de base. La grande partie de ceux ci sont disponibles après avoir listé les assurances. Après cela, on peut voir l’historique des assurances, s’inscrire à une assurance en payant la prime annuelle, payer la prime annuelle d’une assurance et modifier une assurance. Si l’on souhaite modifier une assurance, trois choix s’offrent à nous. Soit on peut annuler une assurance, soit gérer le montant sur une assurance, soit modifier les paramètres d’une assurance. A côté de cela on peut obtenir des informations sur les assurances et introduire un devis. Pour tout les cas importants, une confirmation est demandée avant d’effectuer l’action en question.


\end{document}